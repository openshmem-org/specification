\section*{Incorporating \openshmem{} into Programs}

In this section, we describe how to write a ``Hello World" \openshmem program.
To write a ``Hello World" \openshmem program we need to: 

\begin{itemize}
\item Add the include file shmem.h (for \Clang) or shmem.fh (for \Fortran).
\item Add the initialization call \FUNC{start\_pes}, (line 9) use
  single integer argument, 0, which is ignored~\footnote{The unused
    argument is for compatibility with older SHMEM implementations.}.
\item Use OpenSHMEM calls to query the the total number of PEs (line 10) and PE id (line 11).
\item There is no explicit finalize call, either a return from
  \texttt{main()} (line 13) or an explicit \texttt{exit()} acts as an
  implicit \openshmem finalization.
\item In \openshmem the order in which lines appear
  in the output is not fixed as \ac{PE}s execute asynchronously in parallel.
\end{itemize}

\begin{minipage}{\linewidth}
\vspace{0.1in}
\numberedlisting{label=openshmem-hello,language=OSH+C}{EXAMPLES/hello-openshmem.c}
\outputlisting{language=bash,caption={Expected Output (4
    processors)}}{EXAMPLES/hello-openshmem-c.output}
\vspace{0.1in}
\end{minipage}

\openshmem also has a \Fortran{} API, so for completeness we will now give the
same program written in \Fortran, in listing~\ref{openshmem-hello-f90}:

\begin{minipage}{\linewidth}
\vspace{0.1in}
\numberedlisting{label=openshmem-hello-f90,language=OSH+F}{EXAMPLES/hello-openshmem.f90}
\outputlisting{language=bash,caption={Expected Output (4
    processors)}}{EXAMPLES/hello-openshmem-f90.output}
\vspace{0.1in}
\end{minipage}

The following example shows a more complex \openshmem program that illustrates the use of symmetric data objects.  Note the declaration of the  \VAR{static short target} array and its use as the remote destination in \openshmem short \PUT.  The use of the \VAR{static} keyword results in the \VAR{target} array being symmetric on \ac{PE} \CONST{0} and \ac{PE} \CONST{1}.  Each \ac{PE} is able to transfer data to the \target{} array by simply specifying the local address of the symmetric data object which is to receive the data.  This aids programmability, as the address of the \target{} need not be exchanged with the active side (\ac{PE} \CONST{0}) prior to the RMA (Remote Memory Access) operation.  Conversely, the declaration of the \VAR{short source} array is asymmetric.  Because the \PUT{} handles the references to the \VAR{source} array only on the active (local) side, the asymmetric \source{} object is handled correctly.

\begin{minipage}{\linewidth}
\vspace{0.1in}
\numberedlisting{label=openshmem-hello,language=OSH+C}{EXAMPLES/writing_shmem_example.c}
\outputlisting{language=bash,caption={Expected Output (4
    processors)}}{EXAMPLES/writing_shmem_example.output}
\vspace{0.1in}
\end{minipage}
