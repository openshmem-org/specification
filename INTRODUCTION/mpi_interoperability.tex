\section{\ac{MPI} Interoperability}
\begin{sloppypar} %SP: to prevent constants from running into margins.
\openshmem functions can be used in conjunction with \ac{MPI}
functions  in the same application.  For example, on SGI systems, programs that use both \ac{MPI} and \openshmem functions call \FUNC{MPI\_Init} and \FUNC{MPI\_Finalize} but omit the call to the \FUNC{start\_pes} function.  \openshmem \ac{PE} numbers are equal to the \ac{MPI} rank within the \CONST{MPI\_COMM\_WORLD} environment variable.  Note that this precludes use of \openshmem functions between processes in different \CONST{MPI\_COMM\_WORLD}s. 
\ac{MPI} processes started using the \FUNC{MPI\_Comm\_spawn} function, for
example, cannot use \openshmem functions to communicate with their parent
\ac{MPI} processes.
\end{sloppypar}
On SGI systems \ac{MPI} jobs that use TCP/sockets for inter-host communication, \openshmem functions can be used to communicate with processes running on the same host.  The \FUNC{shmem\_pe\_accessible} function can be used to determine if a remote \ac{PE} is accessible via \openshmem communication from the local \ac{PE}. When running an \ac{MPI} application involving multiple executable files, \openshmem functions can be used to communicate with processes running from the same or different executable files, provided that the communication is limited to symmetric data objects.  On these systems, static memory, such as a \Fortran{} common block or \Clang{} global variable, is symmetric between processes running from the same executable file, but is not symmetric between processes running from different executable files.  Data allocated from the symmetric heap (\FUNC{shmalloc} or \FUNC{shpalloc}) is symmetric across the same or different executable files. The function \FUNC{shmem\_addr\_accessible} can be used to determine if a local address is accessible via \openshmem communication from a remote \ac{PE}.

 Another important feature of these systems is that the \FUNC{shmem\_pe\_accessible} function returns \CONST{TRUE} only if the remote \ac{PE} is a process running from the same executable file as the local PE, indicating that full \openshmem support (static memory and symmetric heap) is available.  When using \openshmem functions within an \ac{MPI} program, the use of \ac{MPI} memory placement environment variables is required when using non-default memory placement options.
