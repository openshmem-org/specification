\section*{History of \openshmem{}}
\begin{description}
\item [{{Cray~SHMEM~(MP-SHMEM,~LC-SHMEM):}}] Cray first introduced
SHMEM in 1993 for its Cray T3D systems. Cray SHMEM was also used in
other models: T3E, PVP and XT series. 
\item [{{SGI~SHMEM~(SGI-SHMEM):}}] Cray Research merged with Silicon
Graphics (SGI) in February 1996. At this point SHMEM was incorporated
into SGI's Message Passing Toolkit (MPT). The platforms supported
were - SGI Irix, Origin and Altix. 
\item [{{Quadrics~SHMEM~(Q-SHMEM):}}] an optimized API for the Quadrics
QsNet interconnect. It included SGI extensions and provided non-blocking
puts and gets. A joint effort from HCS Lab \& Quadrics incorporated
a program profiling interface called PSHMEM that can aid in the execution
analysis of SHMEM programs. 
\end{description}
The success of SHMEM's performance attracted several vendors to provide
implementations (with varying names and features) for their systems.
Some of them include: 
\begin{description}
\item [{{HP~SHMEM:}}] Based on the Quadrics API. It is included in the
UPC product kit. 
\item [{{Cyclops-64~SHMEM~(C64-SHMEM):}}] this SHMEM API supports the
Cyclops-64 architecture. Most of the core features of Cray SHMEM are
available with some additional interfaces specific to the Cyclops-64
architecture. 

\item [{{IBM~SHMEM:}}] An implementation created by IBM intended for
internal use only. 
\item [{{TurboSHMEM:}}] This implementation uses IBM's Low-Level API
(LAPI) technology to obtain optimized one-sided communication for
the put/get operations. This allows applications written with the
SHMEM API to run on IBM platforms with minimal source code changes. 
\item [{{GPSHMEM:}}] This implementation of SHMEM aims at providing full
portability of applications. It is built mostly with Cray T3D components
and functionalities and provides MPI and ARMCI support. This project
is no longer maintained. 
\end{description}
