%
% Copyright (c) 2011, 2012
%   University of Houston System and Oak Ridge National Laboratory.
% 
% All rights reserved.
% 
% Redistribution and use in source and binary forms, with or without
% modification, are permitted provided that the following conditions
% are met:
% 
% o Redistributions of source code must retain the above copyright notice,
%   this list of conditions and the following disclaimer.
% 
% o Redistributions in binary form must reproduce the above copyright
%   notice, this list of conditions and the following disclaimer in the
%   documentation and/or other materials provided with the distribution.
% 
% o Neither the name of the University of Houston System, Oak Ridge
%   National Laboratory nor the names of its contributors may be used to
%   endorse or promote products derived from this software without specific
%   prior written permission.
% 
% THIS SOFTWARE IS PROVIDED BY THE COPYRIGHT HOLDERS AND CONTRIBUTORS
% ``AS IS'' AND ANY EXPRESS OR IMPLIED WARRANTIES, INCLUDING, BUT NOT
% LIMITED TO, THE IMPLIED WARRANTIES OF MERCHANTABILITY AND FITNESS FOR
% A PARTICULAR PURPOSE ARE DISCLAIMED. IN NO EVENT SHALL THE COPYRIGHT
% HOLDER OR CONTRIBUTORS BE LIABLE FOR ANY DIRECT, INDIRECT, INCIDENTAL,
% SPECIAL, EXEMPLARY, OR CONSEQUENTIAL DAMAGES (INCLUDING, BUT NOT LIMITED
% TO, PROCUREMENT OF SUBSTITUTE GOODS OR SERVICES; LOSS OF USE, DATA, OR
% PROFITS; OR BUSINESS INTERRUPTION) HOWEVER CAUSED AND ON ANY THEORY OF
% LIABILITY, WHETHER IN CONTRACT, STRICT LIABILITY, OR TORT (INCLUDING
% NEGLIGENCE OR OTHERWISE) ARISING IN ANY WAY OUT OF THE USE OF THIS
% SOFTWARE, EVEN IF ADVISED OF THE POSSIBILITY OF SUCH DAMAGE.
%


\chapter{Implementation Of The Reference \openshmem Library}

Along with the new specification for \openshmem, there is a reference
library that tracks the development of the specification, and acts as
a testbed for the evolution of the specification.  This chapter
discusses the implementation of the reference library.

\section{Implementation Strategy}

One important goal of the reference implementation is not to be tied
to a particular environment, so it can be used on different platforms
and gain wider exposure for \openshmem.  To this end we have used
GASNet~\cite{gasnet} initially to abstract away from a particular
interconnect and platform, although there is nothing to stop a more
direct approach being used to target specific hardware, as for example
in MVAPICH2-X~\cite{mvapich2-x}.  Internally, the reference
implementation of \openshmem provides private APIs for memory
management, the communications layer, tracing and debugging, and
support for adaptivity to choose things like a barrier algorithm at
run-time.

\subsection{GASNet}

GASNet~\cite{gasnet} (\textbf{G}lobal-\textbf{A}ddress \textbf{S}pace
\textbf{Net}working) is a low-level library that has been used to
implement the runtime of a number of popular HPC PGAS languages, such
as Unified Parallel C~\cite{upc}, Chapel~\cite{chapel} and Co-Array
Fortran~\cite{coarray}.

GASNet manages the start-up of a program and coordinates between the
``nodes'' that make up the program.  It also makes available to all
nodes information about all the other nodes, such as where memory
segments have been allocated.  The details of this are discusssed
below.

GASNet provides an abstraction away from the underlying hardware on a
given platform.  Hardware-specific interfaces are implemented as
modules, called ``conduits'': there are conduits for all the prevalent
communication systems in HPC systems, including networks such as
Ethernet, Infiniband, portals and Myrinet, and one specifically for
shared-memory multiprocessor machines (SMP).  Support for onduits is
detected when GASNet is configured and can further be enabled or
disabled according to requirements.

Three different communication scenarios are supported via ''segments'':

\begin{table}[h]
  \begin{center}
    \caption{GASNet Segment Configurations}
    \begin{tabular}{|p{0.2\textwidth}|p{0.7\textwidth}|}
      \hline 
      Segment Name & Intended for \ldots \tabularnewline
      \hline
      \hline 
      fast & a ``reasonable'' size symmetric segment and
      implementation-specific tuning for fastest data access \tabularnewline
      \hline 
      large & a larger symmetric segment and this may impose some data access penalty \tabularnewline
      \hline 
      everything & the entire process' memory space is available, not just a managed data segment \tabularnewline
      \hline 
    \end{tabular}
  \end{center}
\end{table}

Segments are how GASNet manages the per-node memory areas that are
remotely accessible to other nodes.  Not all conduits are supported
with all 3 segment configurations, which has implications for how
\openshmem's ``symmetric'' variables are handled, and this is discussed
below.

\subsection{How \openshmem uses GASNet}

GASNet provides access to the symmetric memory areas. A memory
management library marshals accesses to these areas during allocation
and freeing of symmetric variables in user code, usually through a
call like \texttt{shmalloc()} or \texttt{shfree()}.

When the program does \texttt{gasnet\_attach()} and asks for segment
information, each PE has access to an array of segments, 1 segment
per PE. Each PE initializes a memory pool within its own segment.
The set up is handled either by GASNet internally (``fast''/''large''
model) or by \openshmem itself (``everything'' model). The table of
segments allows any PE to know the virtual location and size of the
segment belonging to any other PE.

If the platform allows it, GASNet can align all the segments at the
same address, which means that all PEs see the same address for
symmetric variables and there's no address translation.

In the general case though, segments are not aligned (e.g.\ due to a
security measure like process address space randomization by the
OS). However, each PE can see the addresses of the segments of the
other PEs locally, and can therefore do address translation.

Currently alignment is not checked for, so we're coding to the
``worst case scenario''. That just adds a \emph{small} overhead if
the segments are in fact aligned. The library should at some point
introduce code that differentiates between aligned and non-aligned
environments with optimized code for the former case (GASNet provides
a macro you can test against).

\subsubsection{Segment Models}

The library currently has best support for the ``everything''
model. This model allows the entire process space to be addressed
remotely. Communication with dynamically allocated data and with
global data is equally easy.

For the ``fast'' and ``large'' models, only the area of the process
memory managed by GASNet is exposed for remote access. This means
extra support has to be added to handle communication with global
variables, because only the symmetric heap is visible.
This is done via Active Messages.

For the SMP conduit, PSHM support is required to run parallel threaded
programs with \openshmem. This excludes the ``everything'' model (at
least for the architectures to hand).

\subsection{\openshmem Initialization \& Finalization}

The \texttt{start\_pes} call handles setting up the \openshmem runtime,
and eventual shutdown. Shutdown is implicit in user code, there is no
call to do this in SGI SHMEM, so we register an exit handler to be
called when \texttt{main()} exits. (Cray SHMEM has an explicit
finalize call, however, and a proposal for a profiling interface has
suggested introducing this to \openshmem.) The segment exchange is a
target for future optimization: in large programs, the start-up time
will become burdensome due to the large number of address/size
communications. Strategies for avoiding this include lazy
initialization and hierarchical or directory-based lookups.

\subsection{Communications Substrate}

The \openshmem library has been written to sit on top of any
communications library that can provide the required
functionality. Initially we have targetted GASNet.

\subsection{Servicing Communications}

GASNet provides this functionality in some cases. The mainline code
needs to spin on variable waits (e.g.\ shmem\_long\_waituntil) to poll
GASNet, otherwise progress is automatic via a servicer unit.  This is
implemented with a progress thread that polls in a continuous loop.

\subsection{Memory Management}

Initially we tried to use the TLSF~\cite{tlsf} library (as used in the
SiCortex SHMEM implementation) but this proved to have weird
interactions with Open-MPI. Tracking program progress with valgrind
\cite{valgrind} suggested that system memory calls were being
intercepted.

So, following the Chapel~\cite{chapel} lead, we now use the
``dlmalloc''~\cite{dlmalloc} library to manage allocations in the
symmetric memory space.

\subsection{Point-to-point routines}

Point-to-point operations are a thin layer on top of GASNet. The
non-blocking put operations with implicit handles provide a way to
subsequently fence and barrier. However, tracking individual handles
explicitly with a hash table keyed on the address of symmetric
variables may give better performance, and this needs to be looked
into.

The Quadrics extensions that add non-blocking calls into the API
proper have already been requested for the \openshmem development. An
initial attempt at these are already in the library and they pass the
Cray verification tests.

\subsection{Atomic Operations}

Atomic operations include swaps, fetch-and-add and locks (discussed
separately in \ref{sub:Locks}). The first two are handled via GASNet's
Active Messages. Increment was originally layered on top of add
(increment is simply a special case of add) but was rewritten with its
own handlers. The payload for increment can be ever so slightly
smaller than for add since there's no need to pass the value to
add. At large scale, even such a small saving could pay off.

Earlier versions of the implementation had a single handler lock
variable per operation (one for all adds, one for all increments,
\emph{etc.}).  However, there is now a hash table to allocate and
manage per-target-address handler locks. Large-scale atomic
operations, like add-scatters across multiple variables could easily
benefit from this, as the lock granularity then permits concurrent
discrete memory accesses.

\subsection{\label{sub:Locks}Locks}

\openshmem provides routines to claim, release and test global
locks~\footnote{Not to be confused with GASNet's Active Message
  Handler Locks.}.  These can be used for mutual-exclusion
regions. Our implementation is from the Quadrics library, which is a
version of the Mellor-Crummy-Scott
algorithm~\cite{Mellor-Crummey:1991:ASS:103727.103729}.  The locks are
layered on top of \openshmem primitives, so there are no Elan
dependencies.

\subsection{Barrier and broadcast}

The initial version is naive, making the root of the broadcast a
bottleneck.  This is partly intentional, to allow scope to explore
better algorithms and work out how to demonstrate and document the
improvements. We would like to collect some locality information
inside the library to help decide communication order inside these
algorithms: PEs that differ in rank by large amounts are likely to be
further away topologically too, so by sending to more distant PEs
first, we can stagger the network traffic and balance the latencies
better. A proper measurement of ``distance'' is needed here. The
``hwloc'' package~\cite{hwloc} provides a per-system distance metric
in NUMA terms. An extension could e.g.\ just multiply the distance by
some constant when moving off-node to penalize network traffic.

\subsection{Collects}

The collector routines concatenate (parts of) source arrays on an
active set of PEs into a target array on all of those PEs.

\begin{description}
\item[collect] is the general routine in which each participating PE
  can write different amounts of data.
\item[fcollect] is an optimization in which all participating PEs
  \emph{must} contribute the same amount of data. This means we can
  just pre-compute where each PE writes to their targets.
\end{description}

Two approaches were considered for ``collect'':

\begin{enumerate}
\item initial exchange of sizes ``from the left'' so each PE can
  compute its write locations; then same as \texttt{fcollect}
\item wavefront: PEs wait for notification from PEs before them in the
  set (lower numbered). This passes the offsets across the set.
\end{enumerate}

The library uses the ``wavefront''. The ``exchange'' method
potentially generates a network storm as all PEs wait to work out
where to write, then all write at once, whereas the ``wavefront''
staggers the offset notification with a wave of writes moving up the
PE numbers.

\subsection{Reductions}

Reductions coalesce data from a number of PEs into either a single
variable or array on all participating PEs. The coalescing involves
some kind of arithmetic or logic operation (e.g.\ sum, product,
exclusive-or).  Currently probably naive, using gets. A version with
puts that can overlap communication and the computation of the
reduction operation should be more scalable. However, the code is
rather compact and all ops use the same template. A future version of
\openshmem may add user-defined reductions, and in fact the framework
for this is already in place: all that is needed is a specification of
the SHMEM API.

\subsection{Address and PE Accessibility}

\openshmem allows us to test whether PEs are currently reachable, and
whether addresses on remote PEs are addressable. GASNet is used to
``ping'' the remote PE and then we wait for an ``ack'' with a
configurable timeout. Remains to be seen how useful this is, and
whether it can be used for future fault tolerance issues.

\subsection{Tracing Facility}

This library contains \textquotedblleft{}trace
points\textquotedblright{} with categorized messages. These are listed
in section \ref{sec:Environment-Variables}

A high-resolution clock is maintained to timestamp such messages.
Numerically sorting the output on the first field can thus help
understand the order in which events happened.

\subsection{C++}

The C++ interface is basically the C one. There is one point of
contention, namely complex numbers. The SGI documentation refers only
to the use of C99~\cite{c99} ``complex'' modifiers, not to C++'s
\texttt{complex<T>}.  The use of complex number routines
(e.g.\ reductions) in C++ is thus not clearly specified.

\subsection{Fortran}

The Fortran interface is very similar to that of C. The names of
various routines are different to accommodate the various type
differences, e.g.\ \texttt{shmem\_integer\_put()} instead of
\texttt{shmem\_int\_put()}.

The biggest difference is in the symmetric memory management routines.
These have completely different names and parameters compared to the C
interface.

The \openshmem implementation handles Fortran with a very thin wrapper
on top of C. Mostly this involves catching Fortran's pass-by-reference
variables and dereferencing them in the underlying C call.

The main development has been on a CentOS platform with GNU
4.1.2-redhat.  There seem to be some issues with this compilers'
handling of cray-pointers: even the simplest programs (no \openshmem
content at all) produce a segmentation fault. Later versions
(4.5.0 and newer) behave better.

\section{Undefined Behavior}

Many routines are currently specified only in terms of ``correct''
behavior. What happens when something goes wrong is not always
specified.  This section attempts to set out a few of these scenarios
\begin{itemize}
\item put to PE out of range: suppose we do a put to ``right
neighbor'' (\(pe + 1\)). The highest-numbered PE will attempt to
communicate with a PE that does not exist.
\item library not initialized: virtually all \openshmem routines will
have major problems if the library has not been
initialized. Implementations can handle this situation in different
ways.
\end{itemize}

\section{Environment Variables\label{sec:Environment-Variables}}

The behavior of the \openshmem library can be controlled via a number
of environment variables. For SGI compatibility reasons, we support
the ``SMA'' variables:

% force the SGI table now
\clearpage

\begin{table}[!h]
  \begin{center}
    \caption{SGI Environment Variables}
    \begin{tabular}{|l|l|}
      \hline 
      Variable & Function\tabularnewline
      \hline
      \hline 
      \texttt{SMA\_VERSION} & print the library version at start-up\tabularnewline
      \hline 
      \texttt{SMA\_INFO} & print helpful text about all these environment variables\tabularnewline
      \hline 
      \texttt{SMA\_SYMMETRIC\_SIZE} & number of bytes to allocate for symmetric heap\tabularnewline
      \hline 
      \texttt{SMA\_DEBUG} & enable debugging messages\tabularnewline
      \hline
    \end{tabular}
  \end{center}
\end{table}

and our own new ones:

\begin{description}
\item[{\texttt{SHMEM\_LOG\_LEVELS}:}] a comma, space, or semi-colon separated
  list of logging/trace facilities to enable debugging messages. The
  facilities currently include the case-insensitive names:
  
  \begin{table}[!h]
    \begin{center}
      \caption{Logging Facility Names}
      \begin{tabular}{|l|l|}
        \hline 
        Facility & Meaning\tabularnewline
        \hline
        \hline 
        FATAL & something unrecoverable happened, abort\tabularnewline
        \hline

        DEBUG & used for debugging purposes\tabularnewline
        \hline
        INFO & something interesting happened\tabularnewline
        \hline
        SYMBOLS & to inspect the symbol table information\tabularnewline
        \hline
        VERSION & about the library version\tabularnewline
        \hline

        INIT & set-up of the program\tabularnewline
        \hline
        FINALIZE & tear-down of the program\tabularnewline
        \hline
        NOTICE & important event, but non-fatal (see below)\tabularnewline
        \hline
        AUTH & when something is attempted but not allowed\tabularnewline
        \hline
        MEMORY & symmetric memory information\tabularnewline
        \hline
        CACHE & cache flushing operations\tabularnewline
        \hline
        BARRIER & about barrier operations\tabularnewline
        \hline
        BROADCAST & about broadcast operation\tabularnewline
        \hline
        COLLECT & about collect and fcollect operation\tabularnewline
        \hline
        QUIET & tracing network quiet events\tabularnewline
        \hline
        FENCE & tracing network fence events\tabularnewline
        \hline
        REDUCTION & about reduction operations\tabularnewline
        \hline
        LOCK & related to setting, testing and clearing locks\tabularnewline
        \hline
        SERVICE & related to the network service thread\tabularnewline
        \hline
        PROFILING & for the PSHMEM profiling interface\tabularnewline
        \hline
        MODULES & loadable modules for different implementations of routines\tabularnewline
        \hline
      \end{tabular}
    \end{center}
  \end{table}

\item [{\texttt{SHMEM\_LOG\_FILE}:}] a filename to which to write log
  messages.  All PEs append to this file. The default is for all PEs to
  write to standard error. Per-PE log files might be an interesting
  addition.
\item [{\texttt{SHMEM\_SYMMETRIC\_HEAP\_SIZE}:}] the number of bytes
  to allocate for the symmetric heap area. Can scale units with ``K'',
  ``M'' etc. modifiers. The default is 2G.
\item [{\texttt{SHMEM\_BARRIER\_ALGORITHM}:}] the version of the
  barrier to use. The default is ``naive''. Designed to allow people
  to plug other variants in easily and test.
\item [{\texttt{SHMEM\_BARRIER\_ALL\_ALGORITHM}:}] as for
  \texttt{SHMEM\_BARRIER\_ALGORITHM}, but separating these two allows us
  to optimize if e.g.\ hardware has special support for global barriers.
\item [{\texttt{SHMEM\_PE\_ACCESSIBLE\_TIMEOUT}:}] the number of
  seconds to wait for PEs to reply to accessiblity checks. The default
  is 1.0 (i.e.\ may be fractional).  Currently not used.
\end{description}

\section{Alternate collective algorithms}

A module system coupled with the above environment variables allows
for runtime decisions to be made about which algorithm should be used
for different collective routines.  These are installed as dynamic
objects and the selected algorithm is then loaded as needed.  Each
module defines a structure that maps the interface it provides to its
routines.  The choice of algorithm can also be steered through an
optional configuration file (overridden by the environment variables).
The file is

\begin{minipage}{\linewidth}
\begin{lstlisting}[caption={Configuration File Location}]
<installdir>/lib/modules/config
\end{lstlisting}
\end{minipage}

and has a simple

\begin{minipage}{\linewidth}
\begin{lstlisting}[caption={Configuration File Format}]
algorithm = implementation
\end{lstlisting}
\end{minipage}

format, e.g.\

\begin{minipage}{\linewidth}
\begin{lstlisting}[caption={Configuration File Example}]
default = tree
barrier-all = bruck
\end{lstlisting}
\end{minipage}

\section{Compiling and Running Programs}

\subsection{SGI SHMEM}

SHMEM for the SGI Altix family of machines is provided as part of the
Message-Passing Toolkit (MPT) in the ProPack~\cite{ProPack}
suite. Compilation uses a standard C, C++ or Fortran compiler
(e.g.\ GNU~\cite{gcc} or Intel~\cite{intelcomp} suites) and links
against the SMA and MPI libraries:

\begin{lstlisting}[caption={Compiling and Linking SGI SHMEM}]
$ gcc -o prog prog.c -lsma -lmpi
\end{lstlisting}

and a compiled program (C, C++ and Fortran) is run like this via the
MPI launcher:

\begin{lstlisting}[label=oshrun,caption={Running a program on 8 processors with the SGI SHMEM Implementation}]
$ mpirun -np 8 ./prog
\end{lstlisting}

\subsection{\openshmem}

In order to abstract the compilation and launching process for
\openshmem we have provided wrapper programs in the reference
implementation:

\begin{table}[h]
  \begin{center}
    \caption{Wrapper Programs}
    \begin{tabular}{| l | l |}
      \hline
      \texttt{oshcc} & compile and link C programs \tabularnewline
      \hline
      \texttt{oshCC} & compile and link C++ programs \tabularnewline
      \hline
      \texttt{oshfort} & compile and link F77/F90 programs \tabularnewline
      \hline
      \texttt{oshrun} & to launch programs \tabularnewline
      \hline
    \end{tabular}
  \end{center}
\end{table}

The similarity to the style of wrappers found in many MPI
implementations is obvious and intentional.

So to compile with the reference implementation, the wrapper program
can be used as in listing~\ref{oshcc} to compile and link a C program:

\begin{lstlisting}[numbers=none,label=oshcc,caption={Compiling and Linking C with the \openshmem Reference Implementation}]
$ oshcc -O3 -o prog prog.c
\end{lstlisting}

and a compiled program (C, C++ and Fortran) is run like this:

\begin{lstlisting}[label=oshrun,caption={Running a program on 8 processors with the \openshmem Reference Implementation}]
$ oshrun -np 8 ./prog
\end{lstlisting}

\subsection*{Note to Implementers}

There is \emph{no} requirement in \openshmem to provide wrapper
programs like this, it's merely a convenience that helped during the
development of the reference implementation.  It does help
implementations hide the details from programmers to simplify use of
\openshmem, though.

\section{Configuration and Installation}

There is a top-level \texttt{configure} script that is a simplified
version of the GNU autotools. This script will eventually become the
GNU setup and will do lots more feature tests. So the usual procedure
applies:

\begin{lstlisting}
$ /path/to/source/configure [--options...]
$ make
$ make install
\end{lstlisting}

The \texttt{configure} script accepts a
% make sure we get a real double-dash
\texttt{-{}-help} option that lists all the various settings.

\subsection*{Note}
You must run \texttt{configure} from a separate build directory, not
from the source tree.

\section{Accessing \openshmem Headers and Libraries}

There is a useful tool called \texttt{pkg-config}~\cite{pkg-config}
that abstracts access to components of the installation.  With
\texttt{pkg-config} you can ask \openshmem where its header files are,
and which libraries it needs to link programs.

E.g.

\begin{lstlisting}[numbers=none,label=pkg-config-cflags,caption={Finding the \openshmem header directory}]
$ pkg-config --cflags openshmem
-I/opt/openshmem/include
\end{lstlisting}

\begin{lstlisting}[numbers=none,label=pkg-config-libs,caption={Finding the \openshmem libraries}]
$ pkg-config --libs openshmem
-Wl,-rpath,/opt/openshmem/lib -L/opt/openshmem/lib -L/opt/gasnet/fast/lib \
   -L/usr/lib64/gcc/x86_64-suse-linux/4.6 -lopenshmem -lgasnet-ibv-par -libverbs \
  -lpthread -lrt -lgcc -lm
\end{lstlisting}

You can then integrate these settings into your own compilation and linking commands, e.g.

\begin{lstlisting}[numbers=none,label=pkg-config-usage,caption={Compiling with pkg-config}]
$ gcc -c $(pkg-config --cflags openshmem) prog1.c
$ gcc -c $(pkg-config --cflags openshmem) prog2.c
$ gcc -o prog prog1.o prog2.o $(pkg-config --libs openshmem)
\end{lstlisting}
