%
% Copyright (c) 2011, 2012
%   University of Houston System and Oak Ridge National Laboratory.
% 
% All rights reserved.
% 
% Redistribution and use in source and binary forms, with or without
% modification, are permitted provided that the following conditions
% are met:
% 
% o Redistributions of source code must retain the above copyright notice,
%   this list of conditions and the following disclaimer.
% 
% o Redistributions in binary form must reproduce the above copyright
%   notice, this list of conditions and the following disclaimer in the
%   documentation and/or other materials provided with the distribution.
% 
% o Neither the name of the University of Houston System, Oak Ridge
%   National Laboratory nor the names of its contributors may be used to
%   endorse or promote products derived from this software without specific
%   prior written permission.
% 
% THIS SOFTWARE IS PROVIDED BY THE COPYRIGHT HOLDERS AND CONTRIBUTORS
% ``AS IS'' AND ANY EXPRESS OR IMPLIED WARRANTIES, INCLUDING, BUT NOT
% LIMITED TO, THE IMPLIED WARRANTIES OF MERCHANTABILITY AND FITNESS FOR
% A PARTICULAR PURPOSE ARE DISCLAIMED. IN NO EVENT SHALL THE COPYRIGHT
% HOLDER OR CONTRIBUTORS BE LIABLE FOR ANY DIRECT, INDIRECT, INCIDENTAL,
% SPECIAL, EXEMPLARY, OR CONSEQUENTIAL DAMAGES (INCLUDING, BUT NOT LIMITED
% TO, PROCUREMENT OF SUBSTITUTE GOODS OR SERVICES; LOSS OF USE, DATA, OR
% PROFITS; OR BUSINESS INTERRUPTION) HOWEVER CAUSED AND ON ANY THEORY OF
% LIABILITY, WHETHER IN CONTRACT, STRICT LIABILITY, OR TORT (INCLUDING
% NEGLIGENCE OR OTHERWISE) ARISING IN ANY WAY OUT OF THE USE OF THIS
% SOFTWARE, EVEN IF ADVISED OF THE POSSIBILITY OF SUCH DAMAGE.
%


\chapter{The Future of the \openshmem Specification}

Presenting ideas for extending or enhancing the API.  A look at
future technology considerations, e.g.\ how exascale might impact
design decisions.

We have a number of examples of proposed extensions in the SoW.

\section{non-blocking puts and gets}

put/get routines that return per-communication handles to the
caller. The handles can be tested later for completion (present in
Cray and Quadrics SHMEMs, for example)

\section{split-phase collectives}

e.g.\ split-phase shmalloc, barriers, broadcasts

\section{locality}

exposing information about node and cluster topology to the library
and/or its API

\section{regularized namespace}

currently routines are a mixture of ``shmem\_'' prefixes,
``start\_pes'', ``\_my\_pe'' and others. Providing an API with a
consistent naming scheme would be useful, e.g.\

\begin{lstlisting}[language=OSH2+C]
shmem_init (int *argc, char ***argv)

shmem_finalize ()

shmem_malloc ()

shmem_my_pe ()
\end{lstlisting}

\section{Fortran module, C++ namespace}

provide better language support

\section{complex numbers}

The C++ interface is basically the C one. There is one point of
contention, namely complex numbers. The SGI documentation refers only
to the use of C99 ``complex'' modifiers, not to C++'s
\texttt{complex<T>}.  The use of complex number routines (e.g.\
reductions) in C++ is thus not clearly specified.

For example, \texttt{shmem\_sum\_to\_all} has a C variant

\begin{lstlisting}[language=OSH+C]
void shmem_complexd_sum_to_all (double complex *target,
                                double complex *source,
                                ...)
\end{lstlisting}

but in C++, complex doubles would be represented as

\begin{lstlisting}[language=OSH+C]
void shmem_complexd_sum_to_all (complex<double> *target,
                                complex<double> *source,
                                ...)
\end{lstlisting}

\section{User-defined reductions}

\section{thread-safety}

providing thread-safe SHMEM routines that can operate in hybrid
threaded environments, e.g. alongside OpenMP;

\section{fault-tolerance}

\openshmem currently operates under the assumption that there is a
fixed set of PEs available at launch-time and that these remain
accessible for the entire life-time of the program run. Allowing
\openshmem to handle system failures such as nodes becoming
unavailable makes it more attractive for large-scale systems.

2 routines, \texttt{shmem\_pe\_accessible} and
\texttt{shmem\_addr\_accessible} are available in the SGI API.  These
are intended for determining the role of PEs in SGI's particular
environment, which can have mixed-mode MPI and SHMEM programs.
\texttt{shmem\_pe\_accessible} test whether a given rank of the
program is running MPI or SHMEM, and \texttt{shmem\_addr\_accessible}
tests whether a remote address can be reached as a SHMEM PE rather
than an MPI rank.

However, these, or similar, routines could be retained for
fault-tolerance purposes.

\section{Return Types}

Alongside fault-tolerance, many of the routines in the current API are
``void'', that is, they return no value to the calling program.
\texttt{start\_pes} is one example.  Adding return status information
would be useful to allow the program to gracefully trap errors or
other unexpected circumstances, e.g.

\begin{lstlisting}[language=OSH2+C]
int
main (int argc, char *argv[])
{
  int s = shmem_init (&argc, &argv);

  if (s != 0)
    {
      fprintf (stderr, ..., shmem_error (s), ...);
    }

  ...
}
\end{lstlisting}

\section{tools}

that help programming in PGAS environments

\section{where is PGAS going in general?}
