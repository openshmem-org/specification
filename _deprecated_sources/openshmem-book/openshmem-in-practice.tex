%
% Copyright (c) 2011, 2012
%   University of Houston System and Oak Ridge National Laboratory.
% 
% All rights reserved.
% 
% Redistribution and use in source and binary forms, with or without
% modification, are permitted provided that the following conditions
% are met:
% 
% o Redistributions of source code must retain the above copyright notice,
%   this list of conditions and the following disclaimer.
% 
% o Redistributions in binary form must reproduce the above copyright
%   notice, this list of conditions and the following disclaimer in the
%   documentation and/or other materials provided with the distribution.
% 
% o Neither the name of the University of Houston System, Oak Ridge
%   National Laboratory nor the names of its contributors may be used to
%   endorse or promote products derived from this software without specific
%   prior written permission.
% 
% THIS SOFTWARE IS PROVIDED BY THE COPYRIGHT HOLDERS AND CONTRIBUTORS
% ``AS IS'' AND ANY EXPRESS OR IMPLIED WARRANTIES, INCLUDING, BUT NOT
% LIMITED TO, THE IMPLIED WARRANTIES OF MERCHANTABILITY AND FITNESS FOR
% A PARTICULAR PURPOSE ARE DISCLAIMED. IN NO EVENT SHALL THE COPYRIGHT
% HOLDER OR CONTRIBUTORS BE LIABLE FOR ANY DIRECT, INDIRECT, INCIDENTAL,
% SPECIAL, EXEMPLARY, OR CONSEQUENTIAL DAMAGES (INCLUDING, BUT NOT LIMITED
% TO, PROCUREMENT OF SUBSTITUTE GOODS OR SERVICES; LOSS OF USE, DATA, OR
% PROFITS; OR BUSINESS INTERRUPTION) HOWEVER CAUSED AND ON ANY THEORY OF
% LIABILITY, WHETHER IN CONTRACT, STRICT LIABILITY, OR TORT (INCLUDING
% NEGLIGENCE OR OTHERWISE) ARISING IN ANY WAY OUT OF THE USE OF THIS
% SOFTWARE, EVEN IF ADVISED OF THE POSSIBILITY OF SUCH DAMAGE.
%


\chapter{Using \openshmem in the Real World}

\begin{itemize}

\item Some real-life examples of interesting applications written in
  \openshmem.  GUPS, UTS, \ldots?

\item We can take some programs that demonstrate various things that
  tend to be of interest to scientific coders, e.g.\ matrix
  manipulation, FFT and use of SHMEM as a harness around BLAS/LAPACK
  programs.

\item How to add SHMEM to existing programs to parallelize them.

\item How to combine SHMEM with other programming models in hybrid programs.

\item Will want some performance numbers/graphs here to make it
  convincing.

\item Ram has some benchmarky type programs that can go here.

\item Can also talk about hybrid models, e.g.\ \openshmem for
  communication harness, OpenMP intra-node, or \openshmem in purely
  shared memory.

\item What platforms would I want to use \openshmem on?

\end{itemize}
