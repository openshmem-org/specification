\bAPI{SHMEM\_PTR}{Returns  a pointer to a data object on a specified
       \ac{PE}.}
\synC
void *shmem_ptr(void *dest, int pe); %*\synCE    %DO NOT DELETE. THIS LINE IS NOT A COMMENT
\synF
POINTER (PTR, POINTEE)
INTEGER pe
PTR = SHMEM_PTR(dest, pe) %*\synFE   %DO NOT DELETE. THIS LINE IS NOT A COMMENT  

\desB{
\argRow{IN}{dest}{The symmetric data object to be referenced.}
\argRow{IN}{pe}{An integer that indicates the \ac{PE} number on which \dest{} is to
		 be accessed.  If you are using \Fortran, it must be a  default
		 integer value.}
}
{
       \FUNC{shmem\_ptr} returns an address that may be used to directly reference
       \dest{} on the specified \ac{PE}.  This address can be assigned to a pointer.
       After that, ordinary loads and stores to this remote address may be performed.

       When a sequence of loads (gets) and stores (puts) to a data object on a
       remote \ac{PE} does not match the access pattern provided in an \openshmem data
       transfer routine like \FUNC{shmem\_put32} or \FUNC{shmem\_real\_iget}, the
       \FUNC{shmem\_ptr} routine can provide an efficient means to accomplish the
       communication.
}
{
 \desR{
        The return value is a non-NULL address of the \dest{} data object when it is 
        accessible using memory loads and stores in addition to \openshmem{} operations.
        Otherwise, a NULL address is returned.
}
\notesB{
        When calling \FUNC{shmem\_ptr}, \dest{} is the address of 
        the referenced symmetric data object on the calling \ac{PE}.
        }
}

\exampleB{
       \exampleITEM{This  \Fortran{}  program calls \FUNC{shmem\_ptr} and then \ac{PE} 0 writes to the \VAR{BIGD}
       array on \ac{PE} 1:}{./EXAMPLES/shmem_ptr_example.f90}{}
       \exampleITEM{This is the equivalent program written in \Clang:}
       {./EXAMPLES/shmem_ptr_example.c}}{}

\eAPI
