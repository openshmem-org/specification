\bAPI{SHMEM\_ADDR\_ACCESSIBLE}{Determines whether an address is accessible via OpenSHMEM data transfer routines from the specified  remote \ac{PE}.}
%SYNOPSIS
\synC
int shmem_addr_accessible(void *addr, int pe); %*\synCE

\synF
LOGICAL LOG, SHMEM_ADDR_ACCESSIBLE
INTEGER pe
LOG = SHMEM_ADDR_ACCESSIBLE(addr, pe) %*\synFE

%DESCRIPTION

%Arguments
\desB{
       \argRow{IN}{addr}{Data object on the local \ac{PE}.}
       \argRow{IN}{pe}{Integer id of a remote \ac{PE}.}
}
%API Description
{
       \FUNC{shmem\_addr\_accessible} is a query routine that indicates whether a
       local address is accessible via \openshmem routines from the specified
       remote \ac{PE}. 
       
       This routine verifies that the data object is symmetric and accessible
       with respect to a remote \ac{PE} via \openshmem  data  transfer  routines.  The
       specified address \VAR{addr} is a data object on the local \ac{PE}. 
       
       This routine may be particularly useful for hybrid programming with 
       other communication libraries (such as \ac{MPI}) or parallel languages. 
       For example, in SGI Altix series systems, for multiple executable MPI programs
       that use \openshmem routines, it is important to note that static memory,
       such as a \Fortran{} common block or \Clang{} global variable, is symmetric
       between processes running from the same executable file, but is not
       symmetric between processes running from different executable files.
       Data allocated from the symmetric heap (\FUNC{shmem\_malloc} or \FUNC{shpalloc}) is
       symmetric across the same or different executable files.
}
%API Description Table
{
%		\desTB{}
%		{
%				\cRow{}{}
%		}
		%Return Value
		\desR{		
				\CorCpp: The  return value is \CONST{1} if \VAR{addr} is a symmetric data object and
				 accessible via \openshmem routines from the specified remote \ac{PE};
				 otherwise, it is \CONST{0}.

				\Fortran: The return value is \CONST{.TRUE.} if \VAR{addr} is a symmetric data object
				 and accessible via \openshmem routines from the specified remote
				 \ac{PE}; otherwise, it is \CONST{.FALSE.}.
		}
		%NOTES	
		\notesB{None.}
}
%%EXAMPLES
%\exampleB{
%	\exampleITEM
%	{TEST}
%	{./EXAMPLES/shmem_npes_example.c}
%	{}
%}
\eAPI
