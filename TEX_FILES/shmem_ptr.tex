\bAPI{SHMEM\_PTR}{Returns  a pointer  to  a	 data  object  on  a specified
       processing element (PE).}
\synC
void *shmem_ptr(void *target, int pe);
%*\synCE    %DO NOT DELETE. THIS LINE IS NOT A COMMENT
\synF
POINTER (PTR, POINTEE)
INTEGER pe
PTR = SHMEM_PTR(target, pe)
%*\synFE   %DO NOT DELETE. THIS LINE IS NOT A COMMENT  

\desB{
\argRow{IN}{target}{The symmetric data object to be referenced.}
\argRow{IN}{pe}{An integer that indicates the PE number on which target is to
		 be accessed.  If you are using Fortran, it must be a  default
		 integer value.}
}
{
       shmem\_ptr returns an address that may be used  to  directly  reference
       target on the specified PE.  This address can be assigned to a pointer.
       After that, ordinary loads and stores to this  remote  address  may  be
       performed.

       When a sequence of loads (gets) and stores (puts) to a data object on a
       remote PE does not match the access pattern provided in	a OpenSHMEM data
       transfer routine   like  shmem\_put32()  or  shmem\_real\_iget(),  the
       shmem\_ptr function can provide an efficient  means  to  accomplish  the
       communication.
}
{
 \desR{shmem\_ptr returns a pointer to the data object on the specified	remote
       PE. If target is not remotely accessible, a NULL pointer is returned.
}
\notesB{The shmem\_ptr function is available  only  on  systems  where  ordinary
       memory  loads  and  stores  are	used  to  implement OpenSHMEM put and get
       operations. When calling shmem\_ptr, you pass the address on the calling	PE  for a symmetric
       array on the remote PE.}
}

\exampleB{
       \exampleITEM{This  Fortran  program calls shmem\_ptr and then PE 0 writes to the BIGD
       array on PE 1:}{./EXAMPLES/shmem_ptr_example.f90}{}
       \exampleITEM{This is the equivalent program written in C:}
       {./EXAMPLES/shmem_ptr_example.c}}{}

\eAPI
