\bAPI{START\_PES}{ Called at the beginning of an \openshmem program to
initialize the execution environment. (This operation is deprecated and is
provided for backwards compatibility. Implementations must include it, and the
operations should function properly while notifying the user about deprecation
of its use.)}
\synC
void start_pes(int npes); %*\synCE
\synF
CALL START_PES(npes) %*\synFE

% Arguments table. If no arguments you can use \argRow{None}{}{} 
\desB{  
       \argRow{npes}{Unused}{ Should be set to \CONST{0}.}
}
 %API description
 {   
     The \FUNC{start\_pes} routine initializes the \openshmem execution
     environment.  An \openshmem program must call
     \FUNC{start\_pes} before calling any other \openshmem routine.
 }
 %API Description Table. 
{
 %Return Values     
\desR{ None. }
\notesB{ 
    If any other \openshmem call occurs before \FUNC{start\_pes}, the
    behavior is undefined.  Although it is recommended to set \VAR{npes} to
    \CONST{0} for \FUNC{start\_pes}, this is not mandated.  The value is ignored.
    Calling \FUNC{start\_pes} more than once has no subsequent
    effect.
}
}%end of DesB
% Notes. If there are no notes, this field can be left empty.
 \notesB{
    As of \openshmem Specification 1.2 the use of \FUNC{start\_pes} has
    been deprecated. Although \openshmem libraries are required to support the
    call, program developers are encouraged to use \FUNC{shmem\_init} instead.
}
%Example
\exampleB{
%For each example, you can enter it as an item.
\exampleITEMF
    { This is a simple program that calls \FUNC{start\_pes}:}
    {./EXAMPLES/shmem_startpes_example.f90}
    {} 
}
\eAPI 
