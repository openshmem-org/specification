\bAPI{SHMEM\_CSWAP}{Performs an atomic conditional swap to a remote data object.}
\synC   %Synopisis for C API

int shmem_int_cswap(int *target, int cond, int value, int pe);
long shmem_long_cswap(long *target, long cond, long value, int pe);
long shmem_longlong_cswap(long long *target, long long cond, long long value, int pe); %*\synCE    %DO NOT DELETE. THIS LINE IS NOT A COMMENT
\cbstart \synF   %Synopsis for FORTRAN API

INTEGER pe
INTEGER*4 SHMEM_INT4_CSWAP,  cond_i4, value_i4, ires_i4
ires_i4 = SHMEM_INT4_CSWAP(target, cond_i4, value_i4, pe)
INTEGER*8 SHMEM_INT8_CSWAP,  cond_i8, value_i8, ires_i8
ires_i8 = SHMEM_INT8_CSWAP(target, cond_i8, value_i8, pe) %*\synFE   %DO NOT DELETE. THIS LINE IS NOT A COMMENT  
\cbend

% Arguments table. If no arguments you can use \argRow{None}{}{} 
\desB{  
\argRow{OUT}{target}{The remotely accessible integer data object to be updated on
		 the remote \ac{PE}. }
                 % If you are using  C/C++, the data type of
		 % target should match that implied in the SYNOPSIS section.} 
\argRow{IN}{cond}{\VAR{cond} is compared to the remote \VAR{target} value. If 			\VAR{cond} and the
		 remote \VAR{target} are equal, then \VAR{value} is swapped into the
		 remote \VAR{target}. Otherwise, the remote \VAR{target} is unchanged.
		 In either case, the old value of the remote \VAR{target} is
		 returned as the function return value. \VAR{cond} must be of the
		 same data type as \VAR{target}.}
\argRow{IN}{value}{The \VAR{value} to be atomically written to the remote \ac{PE}. 			\VAR{value} must be the same data type as \VAR{target}.}
\argRow{IN}{pe}{An integer that indicates the \ac{PE} number upon which \VAR{target} is to be updated. If you are using \Fortran, it must be a default integer value.}
}
%API description
{  
The conditional swap routines conditionally update a \VAR{target} data object
on an arbitrary \ac{PE} and return the prior contents of the data object in one atomic operation.   
}
%API Description Tabl
{
\hfill \\
\desTB {The  \VAR{target}  and source data objects must conform to certain typing
constraints, which are as follows: } 
{
\cRow{SHMEM\_INT4\_CSWAP}{\CONST{4}-byte integer.}
\cRow{SHMEM\_INT8\_CSWAP}{\CONST{8}-byte integer.}
}
\desR{The contents that had been in the \VAR{target} data object on the remote \ac{PE}
prior to the conditional swap. Data type is the same as the \VAR{target} data type.}
% Notes. If there are no notes, this field can be left empty.
\notesB{None.}
}
%Example
\cbstart \exampleB{
    %For each example, you can enter it as an item.
    \exampleITEM
    {The following call ensures that the first \ac{PE} to execute the
     conditional swap will successfully write its \ac{PE} number to \VAR{race\_winner}
     on \ac{PE} \CONST{0}.}
    {./EXAMPLES/shmem_cswap_example.c}
    {}
} \cbend 	
\eAPI
%%%%%%%%%%%%%%%%%%%%%%%%%%%%%%%%%%%%%%%%%%%%%%%%%%%%%%%%%%%%%%%%%%%%%%%%%%%%%%
%        Performs an atomic conditional swap to a	remote
%        data object.
% 
% SYNOPSIS
%        C or C++:
% 
% 	  int shmem_int_cswap(int *target, int cond, int value, int pe);
% 
% 	  long shmem_long_cswap(long *target, long cond, long value, int pe);
% 
% 	  long	shmem_longlong_cswap(long long *target, long long cond, long long
% 	  value, int pe);
% 
%        Fortran:
% 
% 	  INTEGER pe
% 
% 	  INTEGER(KIND=4) SHMEM_INT4_CSWAP
% 	  ires = SHMEM_INT4_CSWAP(target, cond, value, pe)
% 
% 	  INTEGER(KIND=8) SHMEM_INT8_CSWAP
% 	  ires = SHMEM_INT8_CSWAP(target, cond, value, pe)
% 
% DESCRIPTION
% 
% Arguments
% 
% 
%        OUT	target	 The  remotely accessible integer data object to be updated on
% 		 the remote PE.	 If you are using  C/C++,  the	data  type  of
% 		 target should match that implied in the SYNOPSIS section.  If
% 		 you are using Fortran, it must be of the following type:
% 
% 		 Routine	       Data Type
% 
% 		 SHMEM_INT4_CSWAP      4-byte integer
% 
% 		 SHMEM_INT8_CSWAP      8-byte integer
% 
%        IN	cond	 cond is compared to the remote target value.  If cond and the
% 		 remote	 target	 are  equal,  then  value  is swapped into the
% 		 remote target.	 Otherwise, the remote	target	is  unchanged.
% 		 In  either  case,  the	 old  value  of	 the  remote target is
% 		 returned as the function return value.	 cond must be  of  the
% 		 same data type as target.
% 
%        IN	value	 The  value  to be atomically written to the remote PE.	 value
% 		 must be the same data type as target.
% 
%        IN	pe	 An integer that indicates the PE number upon which target  is
% 		 to  be	 updated.   If	you  are  using	 Fortran, it must be a
% 		 default integer value.
% API Description
% 
%        The conditional swap routines conditionally update a target data object
%        on  an  arbitrary processing element (PE) and return the prior contents
%        of the data object in one  atomic  operation.   
% 
% Return Value
% 
%        The contents that had been in the target data object on the  remote  PE
%        prior to the conditional swap. Data type is the same as the target data type.
% 
% 	
% NOTES
% 
% 
% EXAMPLES
% 
%        The   following	call  ensures  that  the  first	 PE  to	 exectute  the
%        conditional swap will successfully write its PE number  to  race_winner
%        on PE 0.
% 	
% 	\lstinputlisting[language=C]{shmem_cswap_example.c}
