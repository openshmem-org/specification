\bAPI{SHMEM\_N\_PES}{Returns the number of \ac{PE}s running in an application.}
%Synopsis C
\synC
int shmem_n_pes(void);
int _num_pes (void); %*\synCE

%Synopsis F
\synF
INTEGER SHMEM_N_PES, N_PES
N_PES = SHMEM_N_PES()
N_PES = NUM_PES() %*\synFE

%DESCRIPTION

%Arguments
\desB{
	\argRow{None}{}{}
}
%API Description
{
	The function returns the number of \ac{PE}s running the application.
}
%API Description Table.
{
%		\desTB{}
%		{
%				\cRow{}{}
%		}
		%Return Value
		\desR{Integer -  Number of \ac{PE}s running the \openshmem application.}
		%NOTES      
	\notesB{ \cbstart As of \openshmem Specification 1.2 the use of \FUNC{\_num\_pes} has been deprecated. Although \openshmem libraries are required to support the call, application developers are encouraged to use \FUNC{shmem\_n\_pes} instead. \cbend  
}
		}
		%EXAMPLES
\exampleB{ \cbstart
	\exampleITEM
	{The following \FUNC{\_num\_pes} example is for \CorCpp{} programs:}
	{./EXAMPLES/shmem_npes_example.c}
	{}
\cbend }
\eAPI

