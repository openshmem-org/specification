\bAPI{SHMEM\_G}{Transfers one data item from a remote \ac{PE}}
\synC   %Synopisis for C API

char shmem_char_g(char *addr, int pe);
short shmem_short_g(short *addr, int pe);
int shmem_int_g(int *addr, int pe);
long shmem_long_g(long *addr, int pe);
long long  shmem_longlong_g(long long *addr, int pe);
float shmem_float_g(float *addr, int pe);
double shmem_double_g(double *addr, int pe);
long double shmem_longdouble_g(long double *addr, int pe); %*\synCE    %DO NOT DELETE. THIS LINE IS NOT A COMMENT

% Arguments table. If no arguments you can use \argRow{None}{}{} 
\desB{  
    \argRow{IN}{addr}{The remotely accessible array element or scalar data object.}
    \argRow{IN}{pe}{The number of the remote \ac{PE} on which \VAR{addr} resides.}
 }
%API description
{
  These routines provide a very low latency get capability for single elements of most basic types. 
}
%This newline is required 
{
%API Description Table.
\desR{
    %Return Values    
    {Returns a single element of type specified in the synopsis.}
}
% Notes. If there are no notes, this field can be left empty.
\notesB{ None.
 }
} \cbstart
\exampleB{
    \exampleITEM
	{The following \FUNC{shmem\_long\_g} example is for \CorCpp{} programs:}
    {./EXAMPLES/shmem_g_example.c}
    {}
} \cbend
\eAPI

       
       
       
       





