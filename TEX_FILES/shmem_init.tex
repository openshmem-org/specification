\bAPI{SHMEM\_INIT}{A collective operation that allocates and initializes the
resources used by the \openshmem library.}
TESTING RCSLATEX
\synC
void shmem_init(void); %*\synCE    
\synF
CALL SHMEM_INIT() %*\synFE

% Arguments table. If no arguments you can use \argRow{None}{}{} 
\desB{  
    \argRow{None.}{}{}
}
%API description
{   
    \FUNC{shmem\_init} allocates and initializes resources used by the
    \openshmem library. It is a collective operation that all \acp{PE} must
    call before any other \openshmem routine may be called. At the end of the
    \openshmem program which it initialized, the call to \FUNC{shmem\_init}
    must be matched with a call to \FUNC{shmem\_finalize}. After a single
    call to \FUNC{shmem\_init}, a subsequent call to \FUNC{shmem\_init}
    in the same program results in undefined behavior.
}
%API Description Table. 
{
%Return Values     
\desR{ None. }
\notesB{ 
    As of \openshmem Specification 1.2 the use of \FUNC{start\_pes} has been
    deprecated and is replaced with \FUNC{shmem\_init}. While support for
    \FUNC{start\_pes} is still required in \openshmem libraries, users are
    encouraged to use \FUNC{shmem\_init}. Replacing \FUNC{start\_pes} with
    \FUNC{shmem\_init} in \openshmem programs with no further changes is
    possible; there is an implicit \FUNC{shmem\_finalize} at the end of main.
    However, \FUNC{shmem\_init} differs slightly from \FUNC{start\_pes}:
    multiple calls to \FUNC{shmem\_init} within a program results in undefined
    behavior, while in the case of \FUNC{start\_pes}, any subsequent calls to
    \FUNC{start\_pes} after the first one resulted in a no-op.
}
}%end of DesB
%Example
\exampleB{
%For each example, you can enter it as an item.
    \exampleITEMF
    { This is a simple program that calls \FUNC{shmem\_init}:}
    {./EXAMPLES/shmem_init_example.f90}
    {} 
}  	
\eAPI 
