 \bAPI{SHMEM\_LOCK}{Releases, locks, and tests a mutual exclusion memory lock.}
\synC
void shmem_clear_lock(long *lock);
void shmem_set_lock(long *lock);
int shmem_test_lock(long *lock); %*\synCE    %DO NOT DELETE. THIS LINE IS NOT A COMMENT
\synF   %Synopsis for FORTRAN API

INTEGER lock, SHMEM_TEST_LOCK
CALL SHMEM_CLEAR_LOCK(lock)
CALL SHMEM_SET_LOCK(lock)
I = SHMEM_TEST_LOCK(lock) %*\synFE   %DO NOT DELETE. THIS LINE IS NOT A COMMENT  

\desB{
   \argRow{IN}{lock}{	 A symmetric data object that is a scalar variable or an array
		 of  length \CONST{1}.  This data  object  must  be set to \CONST{0} on all
		 \ac{PE}s prior to the first use.  \VAR{lock}  must  be  of type \CONST{long}.  If you are using \Fortran, it must be of default kind.}
}
{
       The \FUNC{shmem\_set\_lock} routine sets a mutual exclusion lock after  waiting
       for  the lock  to be freed by any other \ac{PE} currently holding the lock.
       Waiting \ac{PE}s are assured of getting the lock in a first-come,
       first-served manner.  The \FUNC{shmem\_clear\_lock} routine releases a lock  previously set by \FUNC{shmem\_set\_lock} after ensuring that all local and remote	 stores initiated in the critical region are complete.  The \FUNC{shmem\_test\_lock} function sets a mutual exclusion lock only if it is currently cleared.  By using this function, a \ac{PE} can avoid blocking on a set lock.  If the lock is currently set, the function returns without waiting.  These routines are appropriate for protecting a critical region from simultaneous update by multiple \ac{PE}s.	  
}
{
\desR{
       The \FUNC{shmem\_test\_lock} function returns \CONST{0} if  the lock  was  originally cleared and  this  call was  able  to set the lock.  A value of \CONST{1} is
       returned if the lock had been set and the call returned without waiting
       to set the lock.}
\notesB{
       The term symmetric data object is defined in Introduction. The lock variable should always be initialized to zero and accessed only by the \openshmem locking \ac{API}.
       Changing the value of the lock variable by other means without using the \openshmem \ac{API}, can lead to undefined behavior.             
%       Section 41, there was discussion on the list about putting in language about the opacity of the lock variable after the routines have touched it.  Initialize to zero, then only the API should be allowed to use it, cannot %guarantee any value meaningful to the programmer and any reset could lead to bad things.  Do we want to tighten this up in this version? (e.g. from Brian Barrett)
}
}

\cbstart\exampleB {
\exampleITEM	
 {The following  simple example uses  \FUNC{shmem\_lock}  in a \Clang{} program.}
 {./EXAMPLES/shmem_lock_example.c}{}
} \cbend
\eAPI
