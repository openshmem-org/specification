\apisummary{
    Atomically fetches the value of a remote data object.
}

\begin{apidefinition}

\begin{C11synopsis}
TYPE shmem_atomic_fetch(const TYPE *dest, int pe);
\end{C11synopsis}
where \TYPE{} is one of the extended \ac{AMO} types specified by
Table~\ref{extamotypes}.

\begin{Csynopsis}
TYPE shmem_<TYPENAME>_atomic_fetch(const TYPE *dest, int pe);
\end{Csynopsis}
where \TYPE{} is one of the extended \ac{AMO} types and has a corresponding
\TYPENAME{} specified by Table~\ref{extamotypes}.

\begin{Fsynopsis}
INTEGER pe
INTEGER*4 SHMEM_INT4_FETCH, ires_i4
ires_i4 = SHMEM_INT4_FETCH(dest, pe)
INTEGER*8 SHMEM_INT8_FETCH, ires_i8
ires_i8 = SHMEM_INT8_FETCH(dest, pe)
REAL*4 SHMEM_REAL4_FETCH, res_r4
res_r4 = SHMEM_REAL4_FETCH(dest, pe)
REAL*8 SHMEM_REAL8_FETCH, res_r8
res_r8 = SHMEM_REAL8_FETCH(dest, pe)
\end{Fsynopsis}

\begin{apiarguments}

\apiargument{IN}{dest}{The remotely accessible data object to be fetched from
    the remote \ac{PE}.}
\apiargument{IN}{pe}{An integer that indicates the \ac{PE} number from which
    \VAR{dest} is to be fetched.}

\end{apiarguments}

\apidescription{
    \FUNC{shmem\_atomic\_fetch} performs an atomic fetch operation.
    It returns the contents of the \VAR{dest} as an atomic operation.
}

\apireturnvalues{
    The contents at the \VAR{dest} address on the remote \ac{PE}.
    The data type of the return value is the same as the type of
    the remote data object.
}

\apinotes{
    As of \openshmem[1.4], \FUNC{shmem\_fetch} has been deprecated.
    Its behavior and call signature are identical to the replacement
    interface, \FUNC{shmem\_atomic\_fetch}.
}

\end{apidefinition}
