An \openshmem nonblocking collective operation, like a blocking collective
operation, is a group communication operation among the
participants of the team. All participants of the team are required to call the
collective operation.

\begin{enumerate}

\item Invocation semantics: Upon invocation of a nonblocking collective routine,
the operation is initiated and the routine returns without ensuring completion. All participants of the Team
must call this routine with identical arguments.

\item Collective Types: The nonblocking variants supported include barrier all, alltoall,
and broadcast collectives. Other collective operations such as
reductions, collect, barrier, alltoalls, and sync will not have nonblocking variants.

\item Completion semantics:  \openshmem programs can learn the status of the collective operations
using the \FUNC{shmem\_req\_test} routine. The operation is completed after
at least one call to \FUNC{shmem\_req\_test} or a call to \FUNC{shmem\_req\_wait}.

\item Threads: While using SHMEM\_THREAD\_MULTIPLE, the \openshmem
programs are not allowed to call multiple collective operations on different threads
and the same Team.

\end{enumerate}

Note: Like other nonblocking \openshmem operations, the implementations are
expected to asynchronously progress the collective operations. The guidance on
asynchronous progress is provided in Section \ref{subsec:progress}.
