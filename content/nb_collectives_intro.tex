An OpenSHMEM nonblocking collective operation, like blocking collective
operation, is a group communication operation among the
participants of the team. All participants of the team are required to call the
collective operation.

\begin{enumerate}

\item Invocation semantics: The non-blocking collective routine initializes the
buffers, operation type, reduction type, and posts the collective operation. All
participants of the team should call this routine. The routine returns
immediately after posting the operation. 

\item Collective Types: Currently, only the nonblocking alltoall, broadcast, and reduction collective
operations are supported. The reduction operations supported are defined in the
Table \ref{reducetypes}. 

\item Completion semantics: Upon invocation, the collective operations are
posted and returns immediately. A user can learn the status of the collective operations
using the \FUNC{shmem\_collective\_test} routine and can be completed using
the \FUNC{shmem\_collective\_wait} routine.

\item Threads: While using SHMEM\_THREAD\_MULTIPLE, the \openshmem
programs are allowed to call multiple collective operations on different threads
and the same Team. The collective operations invoked on different threads
are ordered by user-provided tag. When the user does not provide the tag, the
library generates the tag and establishes the order.

\end{enumerate}

Note: Like other nonblocking \openshmem operations, the implementations are
expected to asynchronously progress the collective operations. The guidance on
asynchronous progress is provided in Section \ref{subsec:progress}.



