An \openshmem nonblocking collective operation, like blocking collective
operation, is a group communication operation among the
participants of the team. All participants of the team are required to call the
collective operation.

\begin{enumerate}

\item Invocation semantics: Upon invocation of a collective routine interface,
the operation is posted and returned immediately. All participants of the Team
should call this routine.

\item Collective Types: The nonblocking variants supported include barrier, alltoall,
broadcast, and reduction collectives. Other collective operations such as
collect, barrier, alltoalls, and sync will not have nonblocking variants. The
reduction types supported are defined in Table \ref{teamreducetypes}.

\item Completion semantics:  \openshmem programs can learn the status of the collective operations
using the \FUNC{shmem\_req\_test} routine and can be completed using
the \FUNC{shmem\_req\_wait} routine.

\item Threads: While using SHMEM\_THREAD\_MULTIPLE, the \openshmem
programs are allowed to call multiple collective operations on different threads
and the same Team. The collective operations invoked on different threads
are ordered by a user-provided tag. The user may choose to not order the
collective operations by using the library constant
\CONST{SHMEM\_COLL\_UNORDERED} instead of specifying the tag.

\end{enumerate}

Note: Like other nonblocking \openshmem operations, the implementations are
expected to asynchronously progress the collective operations. The guidance on
asynchronous progress is provided in Section \ref{subsec:progress}.



