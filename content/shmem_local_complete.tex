\apisummary{
    Waits for data to be copied out of the \VAR{source} array on all 
    outstanding non-blocking \OPR{Put} and non-blocking \OPR{put-with-signal} 
    issued by a \ac{PE}.
}

\begin{apidefinition}

\begin{Csynopsis}
void @\FuncDecl{shmem\_local\_complete}@(void);
void @\FuncDecl{shmem\_ctx\_local\_complete}@(shmem_ctx_t ctx);
\end{Csynopsis}

\begin{apiarguments}
    \apiargument{IN}{ctx}{A context handle specifying the context on which to
    perform the operation. When this argument is not provided, the operation is
    performed on the default context.}
\end{apiarguments}

\apidescription{
    The \FUNC{shmem\_local\_complete} routine ensures local completion of all 
    non-blocking \OPR{Put} and non-blocking \OPR{put-with-signal} operations 
    issued by a \ac{PE}. Local completion guarantees the reusability of the
    \VAR{source} buffers associated with a \ac{PE} issuing the operation. 
    Local completion does not guarantee any ordering and/or delivery of 
    completion with any visibility guarantees on all \acp{PE}. 
    Return from \FUNC{shmem\_local\_complete} just guarantees that the data has
    been copied out of the \VAR{source} array on all previously posted
    non-blocking \OPR{Put} and non-blocking \OPR{put-with-signal} operations
    in the local \ac{PE}. Memory ordering routines as supported in
    Section~\ref{subsec:memory_order} is still required to provide
    mechanims to ensure ordering and/or delivery of completions on the
    non-blocking \OPR{Put} and non-blocking \OPR{put-with-signal} operations.
}


\apireturnvalues{
    None.
}

\begin{apiexamples}

\apicexample
    {The following example uses \FUNC{shmem\_quiet} in a \Cstd[11] program: }
    {./example_code/shmem_local_complete_example.c}
    {\FUNC{shmem\_local\_complete} allows reusing the \VAR{source} buffer
    without waiting for the completion and global visibility on target process
    \VAR{tpe}}
\end{apiexamples}

\end{apidefinition}

