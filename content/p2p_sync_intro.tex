The following section discusses \openshmem \acp{API} that provide a mechanism
for synchronization between two \acp{PE} based on the value of a symmetric data
object.
The point-to-point synchronization routines can be used to portably ensure
that memory access operations observe remote updates in the order enforced by
the initiator \ac{PE} using the put-with-signal, \FUNC{shmem\_fence} and
\FUNC{shmem\_quiet} routines.

Where appropriate compiler support is available, \openshmem provides
type-generic point-to-point synchronization interfaces via \Cstd[11] generic
selection. Such type-generic routines are supported for the
``point-to-point synchronization types'' identified in
Table~\ref{p2psynctypes}.

The point-to-point synchronization types include some of the exact-width
integer types defined in \HEADER{stdint.h} by \Cstd[99]~\S7.18.1.1 and
\Cstd[11]~\S7.20.1.1. When the \Cstd translation environment
does not provide exact-width integer types with \HEADER{stdint.h}, an
\openshmem implemementation is not required to provide support for these types.
The \FUNC{shmem\_test\_any} and \FUNC{shmem\_wait\_until\_any} routines
require the \CONST{SIZE\_MAX} macro defined in \HEADER{stdint.h} by
\Cstd[99]~\S7.18.3 and \Cstd[11]~\S7.20.3.

\begin{table}[h]
  \begin{center}
    \begin{tabular}{|l|l|}
      \hline
      \TYPE              & \TYPENAME  \\ \hline
      short              & short      \\ \hline
      int                & int        \\ \hline
      long               & long       \\ \hline
      long long          & longlong   \\ \hline
      unsigned short     & ushort     \\ \hline
      unsigned int       & uint       \\ \hline
      unsigned long      & ulong      \\ \hline
      unsigned long long & ulonglong  \\ \hline
      int32\_t           & int32      \\ \hline
      int64\_t           & int64      \\ \hline
      uint32\_t          & uint32     \\ \hline
      uint64\_t          & uint64     \\ \hline
      size\_t            & size       \\ \hline
      ptrdiff\_t         & ptrdiff    \\ \hline
    \end{tabular}
    \TableCaptionRef{Point-to-Point Synchronization Types and Names}
    \label{p2psynctypes}
  \end{center}
\end{table}

The point-to-point synchronization interface provides named constants whose
values are integer constant expressions that specify the comparison operators
used by \openshmem synchronization routines.
The constant names and associated operations are
presented in Table~\ref{p2p-consts}.  For Fortran, the constant names of
Table~\ref{p2p-consts} shall be identifiers for integer parameters of
default kind corresponding to the associated comparison operation.

\begin{table}[h]
  \begin{center}
    \begin{tabular}{ll}
      \hline
      Constant Name                 & Comparison               \\ \hline
      \LibConstRef{SHMEM\_CMP\_EQ}  & Equal                    \\
      \LibConstRef{SHMEM\_CMP\_NE}  & Not equal                \\
      \LibConstRef{SHMEM\_CMP\_GT}  & Greater than             \\
      \LibConstRef{SHMEM\_CMP\_GE}  & Greater than or equal to \\
      \LibConstRef{SHMEM\_CMP\_LT}  & Less than                \\
      \LibConstRef{SHMEM\_CMP\_LE}  & Less than or equal to    \\ \hline
    \end{tabular}
    \TableCaptionRef{Point-to-Point Comparison Constants}
    \label{p2p-consts}
  \end{center}
\end{table}

\color{ForestGreen}
The point-to-point synchronization interface provides support for passing hints
to specify the \openshmem routine that is expected to update the remotely
accessible data object on a local \ac{PE} on which the synchronization operation
is performed. The hints are named constants whose values are integer constant
expressions. The hint names and the associated \openshmem routines that are
expected to update the remotely accessible data object on the local \ac{PE} are
presented in Table~\ref{p2p-hints}.
\begin{table}[h]
  \begin{center}
    \begin{tabular}{ll}
      \hline
      Hint Name & Associated \openshmem routines \\ \hline
      \LibConstRef{SHMEM\_SIGNAL\_UPDATE}  & Data object used for signalling
      through routines from Sections~\ref{subsec:shmem_put_signal} and
      ~\ref{subsec:shmem_put_signal_nbi} \\ \hline
    \end{tabular}
    \TableCaptionRef{Point-to-Point Hint Constants}
    \label{p2p-hints}
  \end{center}
\end{table}
\color{Black}
