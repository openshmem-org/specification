\apisummary{
  Indicate whether at least one variable within an array of variables on the local \ac{PE} meets a specified test condition.
}

\begin{apidefinition}

\begin{C11synopsis}
size_t @\FuncDecl{shmem\_test\_some}@(TYPE *ivars, size_t nelems, size_t *indices, const int *status,
    int cmp, TYPE cmp_value);
\end{C11synopsis}
where \TYPE{} is one of the point-to-point synchronization types specified by
Table \ref{p2psynctypes}.

\begin{Csynopsis}
size_t @\FuncDecl{shmem\_\FuncParam{TYPENAME}\_test\_some}@(TYPE *ivars, size_t nelems, size_t *indices,
    const int *status, int cmp, TYPE cmp_value);
\end{Csynopsis}
where \TYPE{} is one of the point-to-point synchronization types and has a
corresponding \TYPENAME{} specified by Table \ref{p2psynctypes}.

\begin{apiarguments}

  \apiargument{IN}{ivars}{A pointer to an array of remotely accessible data
    objects.}
  \apiargument{IN}{nelems}{The number of elements in the \VAR{ivars} array.}
  \apiargument{OUT}{indices}{An array of indices of length at least
    \VAR{nelems} into \VAR{ivars} that satisfied the test condition.}
  \apiargument{IN}{status}{An optional mask array of length \VAR{nelems}
    that indicates which elements in \VAR{ivars} are excluded from the test set.}
  \apiargument{IN}{cmp}{A comparison operator from Table~\ref{p2p-consts}
    that compares elements of \VAR{ivars} with \VAR{cmp\_value}.}
  \apiargument{IN}{cmp\_value}{The value to be compared with the objects
    pointed to by \VAR{ivars}.}

\end{apiarguments}

\apidescription{
    The \FUNC{shmem\_test\_some} routine indicates whether at least one entry
    in the test set specified by \VAR{ivars} and \VAR{status} satisfies the
    test condition at the calling \ac{PE}.  The \VAR{ivars} objects at the
    calling \ac{PE} may be updated by an \ac{AMO} performed by a thread located
    within the calling \ac{PE} or within another \ac{PE}.
    This routine does not block and returns zero if
    no entries in \VAR{ivars} satisfied the test condition.  This routine
    compares each element of the \VAR{ivars} array in the test set with the
    value \VAR{cmp\_value} according to the comparison operator \VAR{cmp} at
    the calling \ac{PE}.  This routine tests all elements of \VAR{ivars} in the
    test set at least once, and the order in which the elements are tested is
    unspecified.  If an entry $i$ in \VAR{ivars} within the test set satisfies
    the test condition, a series of calls to \FUNC{shmem\_test\_some} must
    eventually return $i$.

    Upon return, the \VAR{indices} array contains the indices of the elements
    in the test set that satisfied the test condition during the call to
    \FUNC{shmem\_test\_some}.  The return value of \FUNC{shmem\_test\_some} is
    equal to the total number of these satisfied elements.  If the return value
    is $N$, then the first $N$ elements of the \VAR{indices} array contain
    those unique indices that satisfied the test condition.
    These first $N$ elements of \VAR{indices} may be unordered with respect to
    the corresponding indices of \VAR{ivars}.
    The array pointed
    to by \VAR{indices} must be at least \VAR{nelems} long.
    If an entry $i$ in \VAR{ivars} within the test set satisfies the test
    condition, a series of calls to \FUNC{shmem\_test\_some} must eventually
    include $i$ in the \VAR{indices} array.

    The optional \VAR{status} is a mask array of length \VAR{nelems} where each element
    corresponds to the respective element in \VAR{ivars} and indicates whether
    the element is excluded from the test set.  Elements of \VAR{status} set to
    0 will be included in the test set, and elements set to 1 will be ignored.  If all
    elements in \VAR{status} are set to 1 or \VAR{nelems} is 0, the test set is
    empty and this routine returns 0.  If \VAR{status} is a null pointer, it is ignored and all
    elements in \VAR{ivars} are included in the test set.  The \VAR{ivars},
    \VAR{indices}, and \VAR{status} arrays must not overlap in memory.

    Implementations must ensure that \FUNC{shmem\_test\_some} does not return
    indices before the updates of the memory indicated by the corresponding
    \VAR{ivars} elements are fully complete.
}

\apireturnvalues{
    \FUNC{shmem\_test\_some} returns the number of indices returned in
    the \VAR{indices} array. If the test set is empty, this routine returns 0.
}

\apinotes{
  None.
}

\begin{apiexamples}
  \apicexample
      {The following \Cstd[11] example demonstrates the use of
      \FUNC{shmem\_test\_some} to process a simple all-to-all transfer of N
      data elements via a sum reduction, while potentially overlapping
      communication with computation.  This pattern is similar to the
      \FUNC{shmem\_test\_any} example above, but each while loop iteration may
      process more than one data item.}
      {./example_code/shmem_test_some_example.c}
      {}
\end{apiexamples}

\end{apidefinition}
