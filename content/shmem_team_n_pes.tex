\apisummary{
    shmem\_team\_n\_pes returns the total number of \acp{PE} in the provided team.
}

\begin{apidefinition}

\begin{Csynopsis}
int @\FuncDecl{shmemx\_team\_n\_pes}@(shmem_team_t newteam);
\end{Csynopsis}

\begin{apiarguments}
\apiargument{IN}{newteam}{A valid SHMEM team handle.}
\end{apiarguments}

\apidescription{
The shmemx\_team\_n\_pes function returns the number of processes in the
team. This will always be a value between 1 and the total number of
\acp{PE}. For the team SHMEM\_TEAM\_WORLD, this will return shmem\_n\_pes.
Every team must have a least one member. All processes in the team
will get back the same value for the team size.

Error checking will be done to ensure a valid team handle is provided.
All errors are considered fatal and will result in the job aborting
with an informative error message.
}

\apireturnvalues{
Total number of \acp{PE} in the provided team.
}

\apinotes{
By default, SHMEM creates two predefined teams that will be available
for use once the routine start\_pes has been called. These teams can be
referenced in the application by the constants SHMEM\_TEAM\_WORLD and
SHMEM\_TEAM\_NODE. Every \ac{PE}process is a member of the SHMEM\_TEAM\_WORLD
team, and its rank in SHMEM\_TEAM\_WORLD corresponds to the value of its
global \ac{PE}rank. The SHMEM\_TEAM\_NODE team only contains the set of \acp{PE}
that reside on the same node as the current PE.
}

\end{apidefinition}
