\apisummary{
    Indicate whether a variable on the local \ac{PE} meets the specified condition.
}

\begin{apidefinition}

\begin{C11synopsis}
int @\FuncDecl{shmem\_test}@(TYPE *ivar, int cmp, TYPE cmp_value);
\end{C11synopsis}
where \TYPE{} is one of the point-to-point synchronization types specified by
Table \ref{p2psynctypes}.

\begin{Csynopsis}
int @\FuncDecl{shmem\_\FuncParam{TYPENAME}\_test}@(TYPE *ivar, int cmp, TYPE cmp_value);
\end{Csynopsis}
where \TYPE{} is one of the point-to-point synchronization types and has a
corresponding \TYPENAME{} specified by Table \ref{p2psynctypes}.

\begin{apiarguments}

  \apiargument{IN}{ivar}{A pointer to a remotely accessible data object.}
  \apiargument{IN}{cmp}{The comparison operator that compares \VAR{ivar} with
    \VAR{cmp\_value} \color{ForestGreen} and it may be used with point-to-point
    hint constants by combining them with a bitwise OR operation.\color{Black}}
  \apiargument{IN}{cmp\_value}{The value against which the object pointed to
    by \VAR{ivar} will be compared.}

\end{apiarguments}

\apidescription{
  \FUNC{shmem\_test} tests the numeric comparison of the symmetric object
  pointed to by \VAR{ivar} with the value \VAR{cmp\_value} according to the
  comparison operator \VAR{cmp}.
}

\apireturnvalues{
  \FUNC{shmem\_test} returns 1 if the comparison of the symmetric object
  pointed to by \VAR{ivar} with the value \VAR{cmp\_value} according to the
  comparison operator \VAR{cmp} evaluates to true; otherwise, it returns 0.
}

\apinotes{
  None.
}

\begin{apiexamples}
  \apicexample
      {The following example demonstrates the use of \FUNC{shmem\_test} to
        wait on an array of symmetric objects and return the index of an
        element that satisfies the specified condition.}
      {./example_code/shmem_test_example1.c}
      {}
\end{apiexamples}

\end{apidefinition}
