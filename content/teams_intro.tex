The \acp{PE} in an \openshmem program communicate using either
point-to-point routines that specify the \ac{PE} number of the target
\ac{PE} or collective routines that operate over some predefined
set of \acp{PE}. Teams in \openshmem allow programs to group subsets
of \acp{PE} for communications. Collective communications operate on
teams objects across the \acp{PE} in the team. Point-to-point routines
can make use of team based renumbering of \acp{PE} by utilizing team
based contexts or \ac{PE} number translation.

An \openshmem team is a set of \acp{PE} defined by calling a specific team
split routine with a parent team argument and other arguments to
specify how the parent team is to be split into one or more new teams.
Any team created by a \FUNC{shmem\_team\_split\_*} routine can subsequently
be used as the parent team for further calls to team split routines.
A team persists and can be used for team-based routine calls
until it is destroyed by \FUNC{shmem\_team\_destroy}.

Every team must have a least one member. Any attempt to create a team over an
empty set of \acp{PE} will result in no new team being created.

\subsubsection*{Team Handles and Predefined Teams}

A ``team handle'' is an opaque object with type \CTYPE{shmem\_team\_t} that is used
to reference a defined team.  Team handles are created by one of the team split
routines and destroyed by the team destroy routine. Team handles have local
semantics only. That is, team handles should not be stored in shared variables
and used across other \acp{PE}. Doing so will result in undefined behavior.

A special team handle value, \LibConstRef{SHMEM\_TEAM\_INVALID}, may be used to
indicate that a returned team handle is not valid. This value can be tested
against to check for successful split operations and can be assigned to user
declared team handles as a sentinel value.

By default, \openshmem creates predefined teams that will be available
for use once the routine \FUNC{shmem\_init} has been called. See
Section~\ref{subsec:library_handles} for a description of all predefined team handles
provided by \openshmem. Predefined \CTYPE{shmem\_team\_t} handles can be used as
the parent team when creating new \openshmem teams.

Every \ac{PE} is a member of the default team, which may be referenced
through the team handle \LibHandleRef{SHMEM\_TEAM\_WORLD}.
The \ac{PE} number in the default team is equal to the
value of its \ac{PE} number as returned by \FUNC{shmem\_my\_pe}.

\subsubsection*{Team Objects and Multithreading Within a \ac{PE}}

Team handles are passed as arguments to a variety of \openshmem routines,
including collective routines (see Section~\ref{subsec:coll}), include team
creation routines.  While \openshmem routines are thread-safe as
per threading model (see section \ref{subsec:thread_support}),\openshmem
teams objects are not themselves thread-safe. It is the responsibility
of the application to ensure that there are no simultaneous collective
routines operating on the same \openshmem team on a given \ac{PE}.

\subsubsection*{Team Objects and Collective Ordering across \acp{PE}}

In \openshmem, a team object encapsulates resources uses to communicate
between \acp{PE} in collective operations. When calling multiple subsequent
collective operations on a team, the collective operations -- along with any
relevant team based resources -- are matched across the \acp{PE} in the team
based on ordering of collective routine calls. It is the responsibility
of the application to ensure a consistent ordering of collective routine calls
across all \acp{PE} in a team.

There is no need for explicit synchronization between subsequent calls
to collective routines across the team, except in the special case discussed
below for team creation of overlapping child teams from a common parent team.

A full discussion of collective semantics follows in Section~\ref{subsec:coll}.

\subsubsection*{Team Creation}

Team creation is a collective operation on the parent team object. New teams
result from a \FUNC{shmem\_team\_split\_*} routine, which takes a parent team
and other arguments and produces new teams that are a subset of the parent
team. Teams that are created by a \FUNC{shmem\_team\_split\_*} routine may be
provided a configuration argument that specifies attributes of each new team.
This configuration argument is of type \CTYPE{shmem\_team\_config\_t}, which
is detailed further in Section~\ref{subsec:shmem_team_config_t}.

As with any collective routine on a team, the program must ensure that there
are no simultaneous split operations occurring on the same parent team on a
given \ac{PE}, i.e. in separate threads.

As with any collective routine on a team, team creation is matched across PEs based
on ordering. So, team creation events must occur in the same order on all \acp{PE}
in the parent team. Additionally, there must not be team creation
operations from the same parent team simultaneously occurring that involve
the same \acp{PE} in any resulting child teams.

\begin{itemize}
\item[] The following rule of practice will avoid any conflicts on team
object resources during team creation:
\item[] \emph{When a parent team is split multiple times, and the resulting child teams
have overlapping membership, the program must call the \FUNC{shmem\_team\_sync}
routine on the parent team between subsequent calls to split routines.}
\end{itemize}

Upon completion of a team creation operation, any resulting child teams will be
immediately usable for any team-based operations, including creating new child teams,
without any intervening synchronization.
