\apisummary{
    Returns the number of \acp{PE} running in a program.
}
\index{SHMEM\_N\_PES}

\begin{apidefinition}

\begin{Csynopsis}
int shmem_n_pes(void);
\end{Csynopsis}

\begin{Fsynopsis}
INTEGER SHMEM_N_PES, N_PES
N_PES = SHMEM_N_PES()
\end{Fsynopsis}

\begin{apiarguments}
    \apiargument{None.}{}{}
\end{apiarguments}

\apidescription{
    The routine returns the number of \acp{PE} running in the program.
}

\apireturnvalues{
    Integer -  Number of \acp{PE} running in the \openshmem program.
}

\apinotes{
    As of \openshmem Specification 1.2 the use of \FUNC{\_num\_pes} has been
    deprecated. Although \openshmem libraries are required to support the call,
    users are encouraged to use \FUNC{shmem\_n\_pes} instead.  The behavior and
    signature  of the routine \FUNC{shmem\_n\_pes} remains unchanged from the
    deprecated \FUNC{\_num\_pes} version.
}

\begin{apiexamples}

\apicexample
	 {The following \FUNC{shmem\_my\_pe} and \FUNC{shmem\_n\_pes} example is for
	  \CorCpp{} programs:}
    {./example_code/shmem_npes_example.c}
    {}

\end{apiexamples}

\end{apidefinition}
