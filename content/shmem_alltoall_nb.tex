\apisummary{
  Exchanges a fixed amount of contiguous data blocks between all pairs
  of \acp{PE} participating in the collective routine.
}

\begin{apidefinition}

%% C11
\begin{C11synopsis}
int @\FuncDecl{shmem\_alltoall\_nb}@(shmem_team_t team, TYPE *dest, const TYPE
*source, size_t nelems, shmem_req_h *request);
\end{C11synopsis}
where \TYPE{} is one of the standard \ac{RMA} types specified by Table \ref{stdrmatypes}.

\begin{Csynopsis}
\end{Csynopsis}
\begin{CsynopsisCol}
int @\FuncDecl{shmem\_\FuncParam{TYPENAME}\_alltoall\_nb}@(shmem_team_t team,
TYPE *dest, const TYPE *source, size_t nelems, shmem_req_h *request);
\end{CsynopsisCol}
where \TYPE{} is one of the standard \ac{RMA} types and has a corresponding \TYPENAME{} specified by Table \ref{stdrmatypes}.

\begin{CsynopsisCol}
int @\FuncDecl{shmem\_alltoallmem\_nb}@(shmem_team_t team, void *dest, const
void *source, size_t nelems, shmem_req_h *request);
\end{CsynopsisCol}

\begin{apiarguments}

\apiargument{IN}{team}{A valid \openshmem team handle to a team.}%

\apiargument{OUT}{dest}{Symmetric address of a data object large enough to receive
    the combined total of \VAR{nelems} elements from each \ac{PE} in the
    team.
    The type of \dest{} should match that implied in the SYNOPSIS section.}
\apiargument{IN}{source}{Symmetric address of a data object that contains \VAR{nelems}
    elements of data for each \ac{PE} in the team, ordered according to
    destination \ac{PE}.
    The type of \source{} should match that implied in the SYNOPSIS section.}
\apiargument{IN}{nelems}{
  The number of elements to exchange for each \ac{PE}.
  For \FUNC{shmem\_alltoallmem\_nb} it represents bytes.
}
\apiargument{OUT}{request}{An opaque request handle identifying the collective
operation.}

\end{apiarguments}

\apidescription{
    The \FUNC{shmem\_alltoall\_nb} routines are collective routines. All
    \acp{PE} in the provided team must participate in the collective. If
    \VAR{team} compares equal to \LibConstRef{SHMEM\_TEAM\_INVALID} or is
    otherwise invalid, the behavior is undefined.

    {\bf Invocation and completion}: A call to the nonblocking alltoall routine initiates the operation and returns
    immediately without necessarily completing the operation. On success,
    an opaque request handle is created and returned. The
    operation is completed after a call to \FUNC{shmem\_req\_test} or
    a call to \FUNC{shmem\_req\_wait}. When the operation is complete, the request handle
    is deallocated and cannot be reused.

    Though nonblocking alltoall varies in invocation and completion semantics
    when compared to blocking alltoall, the data exchange semantics are similar.

    {\bf Data exchange semantics}:
    In this routine, each \ac{PE}
    participating in the operation exchanges \VAR{nelems} data elements
    with all other \acp{PE} participating in the operation.
    The size of a data element is:
    \begin{itemize}
    \item 8 bits for \FUNC{shmem\_alltoallmem\_nb}
    \item \FUNC{sizeof}(\TYPE{}) for alltoall routines taking typed \VAR{source} and \VAR{dest}
    \end{itemize}

    The data being sent and received are
    stored in a contiguous symmetric data object. The total size of each \ac{PE}'s
    \VAR{source} object and \VAR{dest} object is \VAR{nelems} times the size of
    an element
    times \VAR{N}, where \VAR{N} equals the number of \acp{PE} participating
    in the operation.
    The \VAR{source} object contains \VAR{N} blocks of data
    (where the size of each block is defined by \VAR{nelems}) and each block of data
    is sent to a different \ac{PE}.

    The same \dest{} and \source{}
    arrays, and same value for nelems
    must be passed by all \acp{PE} that participate in the collective.

    Given a \ac{PE} \VAR{i} that is the \kth \ac{PE}
    participating in the operation and a \ac{PE}
    \VAR{j} that is the \lth \ac{PE}
    participating in the operation,

    \ac{PE} \VAR{i} sends the \lth block of its \VAR{source} object to
    the \kth block of
    the \VAR{dest} object of \ac{PE} \VAR{j}.

    
    Like data exchange semantics, the entry and completion
    criteria of blocking and nonblocking alltoall are similar.

    {\bf Entry criteria}: Before any \ac{PE} calls a \FUNC{shmem\_alltoall\_nb} routine,
    the following condition must be ensured:
    \begin{itemize}
    \item The \VAR{dest} data object on all \acp{PE} in the team is
      ready to accept the \FUNC{shmem\_alltoall\_nb} data.
    \end{itemize}
    Otherwise, the behavior is undefined.

    {\bf Completion criteria}: Upon completion, the following is true for
    the local PE:
    \begin{itemize}
    \item Its \VAR{dest} symmetric data object is completely updated and
    the data has been copied out of the \VAR{source} data object.
    \end{itemize}
}

\apireturnvalues{
    Zero on successful local completion. Nonzero otherwise.
}

\end{apidefinition}

