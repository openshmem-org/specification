\openshmem \emph{sessions} provide a mechanism for applications to inform the
\openshmem library of an upcoming sequence of communication routines that
exhibit suitable patterns for runtime optimizations.
A session is associated with a specific \openshmem communication context
(Section~\ref{sec:ctx}), and it indicates the beginning and ending of
communication phases on that context.
The \FUNC{shmem\_session\_start} routine indicates the beginning of a session,
and the \FUNC{shmem\_session\_stop} routine indicates the end of a session.
The \LibConstRef{SHMEM\_SESSION\_*} options (Table~\ref{session_opts}) indicate
which patterns of \openshmem RMA and AMO routines will occur within a session.
These options serve only as \textit{hints} to the library; it is up to the
implementation whether or not to apply any optimizations within a session.

Sessions do not affect the completion or ordering semantics of any \openshmem
routines in the program.
For this reason, routines such as non-blocking RMAs, non-blocking AMOs,
non-blocking \OPR{put-with-signal}, blocking scalar \OPR{puts}, and blocking
non-fetching AMOs are viable candidates for optimizations.
Other routines, such as blocking non-scalar \OPR{puts} and \OPR{gets}, blocking
fetching AMOs, blocking scalar \OPR{gets}, and the memory ordering routines
might require the library to enforce remote completion, reducing the potential
benefit of session optimizations.
Because sessions are associated with an \openshmem communication context,
routines not performed on a communication context (like collective routines)
are ineligible for session hints.
