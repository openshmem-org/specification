\openSHMEM \emph{Sessions} provide a mechanism for applications to apply property to
the \openshmem library that a series of operations on a communication context.


The \emph{sessions routines} provide a mechanism for an application to
indicate to the \openshmem
library that it intends to issue a series of operations on a
communication context optimization.

The \FUNC{shmem\_bundle\_start} routine indicates the beginning of a bundling
phase, and the \FUNC{shmem\_bundle\_stop} routine indicates the end of a
bundling phase.

The bundling routines are \textit{hints} to the \openshmem library\newtext{;}\oldtext{,}
\oldtext{and} they do not affect the completion or ordering semantics of any \openshmem
routines in the program.
For this reason, routines such as non-blocking RMAs, non-blocking AMOs,
non-blocking \OPR{put-with-signal}, blocking scalar \OPR{puts}, and blocking
non-fetching AMOs are viable candidates for bundling optimizations.
Other routines, such as blocking non-scalar \OPR{puts} and \OPR{gets}, blocking
fetching AMOs, blocking scalar \OPR{gets}, and the memory ordering routines
might require the library to enforce remote completion, reducing the potential
benefit of bundling optimizations.
Because bundling is performed with respect to an \openshmem communication
context, routines not performed on a communication context \newtext{(like collective routines)} are ineligible for
bundling optimization.
