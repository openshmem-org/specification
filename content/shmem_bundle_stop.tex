\apisummary{
    Stop a communication bundle.
}

\begin{apidefinition}

\begin{Csynopsis}
void @\FuncDecl{shmem\_bundle\_stop}@(void);
void @\FuncDecl{shmem\_ctx\_bundle\_stop}@(shmem_ctx_t ctx);
\end{Csynopsis}

\begin{apiarguments}
    \apiargument{IN}{ctx}{A context handle specifying the context on which to
    perform the optimization. When this argument is not provided, the
    optimization is performed on the default context.}
\end{apiarguments}

\apidescription{
    The \FUNC{shmem\_bundle\_stop} routine provides a hint to the \openshmem
    library to stop applying bundling-related optimizations.
}

\apireturnvalues{
    None.
}

\begin{apiexamples}

\apicexample
    {The following example demonstrates the usage of
    \FUNC{shmem\_bundle\_start} and \FUNC{shmem\_bundle\_stop} with a loop of
    random atomic non-fetching XOR updates to a distributed table, similar to
    the Giga-updates per second (GUPS) microbenchmark
    \footnote{http://icl.cs.utk.edu/projectsfiles/hpcc/RandomAccess/}.}
    {./example_code/shmem_bundle_example.c}
    {}
\end{apiexamples}

\end{apidefinition}

