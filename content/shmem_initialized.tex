\apisummary{
  Indicates whether the \openshmem library has been initialized.
}

\begin{apidefinition}

\begin{Csynopsis}
int @\FuncDecl{shmem\_initialized}@(void);
\end{Csynopsis}

\begin{apiarguments}
  \apiargument{None}{}{}
\end{apiarguments}

\apidescription{
  The \FUNC{shmem\_initialized} routine returns a value indicating
  whether the \openshmem library has been initialized (i.e, a call to
  either \FUNC{shmem\_init} or \FUNC{shmem\_init\_thread} has
  completed successfully).
  This routine may be called at any point in an \openshmem program,
  including before \FUNC{shmem\_init[\_thread]} and after
  \FUNC{shmem\_finalize}.
  This routine may be called by any thread of execution in the
  \ac{PE}, independent of the level of thread support provided by the
  implementation.
}

\apireturnvalues{
  Returns \CONST{1} if the \openshmem library has been initialized;
  otherwise, returns \CONST{0}.
}

\apinotes{
  Although \FUNC{shmem\_initialized} is thread-safe, its return value
  is not a sufficient guard to prevent multiple threads from racing to
  initialize the \openshmem library concurrently, as
  \FUNC{shmem\_initialized} may return \CONST{0} to one thread while
  library initialization is in progress due to a call from another
  thread.  Applications must ensure that only one call to
  \FUNC{shmem\_init[\_thread]} is made to initialize the \openshmem
  library.
}

\end{apidefinition}
