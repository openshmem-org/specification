\apisummary{
    Waits for completion of all outstanding memory store, blocking
    \PUT{}, \ac{AMO}, and \emph{put-with-signal}, as well as
    nonblocking \PUT{}, \emph{put-with-signal}, and \GET{} routines
    to symmetric data objects issued by the calling \ac{PE} at the target
    \acp{PE}.
}

\begin{apidefinition}

\begin{Csynopsis}
void @\FuncDecl{shmem\_pe\_quiet}@(const int *target_pes, size_t npes);
void @\FuncDecl{shmem\_ctx\_pe\_quiet}@(shmem_ctx_t ctx, const int *target_pes, size_t npes);
\end{Csynopsis}

\begin{apiarguments}
    \apiargument{IN}{ctx}{A context handle specifying the context on which to perform the operation.
        When this argument is not provided, the operation is performed on
        the default context.}
  \apiargument{IN}{target\_pes}{Address of target \ac{PE} array where the
    operations need to be completed}
  \apiargument{IN}{npes}{The number of \acp{PE} in the target \ac{PE} array}

\end{apiarguments}

\apidescription{
    The \FUNC{shmem\_pe\_quiet} ensures completion of memory store, blocking
    \PUT{}, \ac{AMO}, and \emph{put-with-signal}, as well as  nonblocking
    \PUT{}, \emph{put-with-signal}, and \GET{} routines on the symmetric data
    objects issued by the calling \ac{PE} to the target \acp{PE} and on the
    given context. If \VAR{npes} is set to 0, the \VAR{target\_pes} is ignored
    and the routine returns immediately.

  The completion and visibility semantics of these operations are the same as the
  \FUNC{shmem\_quiet} routine. However, it applies only to the target
    \acp{PE}, i.e., the operations to the target \acp{PE} are guaranteed to be
    complete and visible to all \acp{PE} when \FUNC{shmem\_pe\_quiet} returns.
}
\apireturnvalues{
    None.
}

\apinotes{
    On certain platforms, when \FUNC{shmem\_pe\_quiet} is invoked on a set of
    \acp{PE}, the performance might be equivalent to \FUNC{shmem\_quiet}.
}
\end{apidefinition}
