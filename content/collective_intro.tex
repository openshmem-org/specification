\emph{Collective Routines} are defined as communication or synchronization operations on a 
group of \acp{PE} called an active set. \index{Active Set} 
The collective routines require all \acp{PE} in the active set to simultaneously call the
routine.  A \ac{PE} that is not part of the active set calling the collective
routines results in an undefined behavior.  All collective routines have an
active set as an input parameter except \barrierall. The \barrierall is
called by all \acp{PE} of the \openshmem program. 

The active set is defined by the arguments \VAR{PE\_start}, \VAR{logPE\_stride},
and \VAR{PE\_size}.  \VAR{PE\_start} is the starting \ac{PE} number, a log (base
2) of \VAR{logPE\_stride} is the stride between \acp{PE}, and \VAR{PE\_size} is
the number of \acp{PE} participating in the active set.  All \acp{PE}
participating in the collective routines provide the same values for these
arguments. 
 
Another argument important to collective routines is \VAR{pSync}, which is a
symmetric work array.  All \acp{PE} participating in a collective must pass the
same \VAR{pSync} array.  On completion of a collective call, the \VAR{pSync} is
restored to its original contents.  The user is permitted to reuse a \VAR{pSync}
array if all previous collective routines using the \VAR{pSync} array have been
completed by all participating \acp{PE}.  One can use a synchronization
collective routine such as \barrier to ensure completion of previous collective
routines. The \FUNC{shmem\_barrier} routine allows the same \VAR{pSync} array to
be used on consecutive calls as long as the \ac{PE} active set does not change. 

All collective routines defined in the specification are blocking.  The
collective routines return on completion.  The collective routines defined in
the \openshmem specification are:
\index{SHMEM\_BROADCAST}
\index{SHMEM\_BARRIER}
\index{SHMEM\_BARRIERALL}
\index{SHMEM\_COLLECT}
\index{SHMEM\_FCOLLECT}

\begin{itemize}
\item[] \broadcast 
\item[] \barrier
\item[] \barrierall
\item[] \collect
\item[] \fcollect
\item[] \reduction
\item[] \alltoall
\item[] \alltoalls
\index{SHMEM\_ALLTOALL}
\index{SHMEM\_ALLTOALLS}
\end{itemize} 
