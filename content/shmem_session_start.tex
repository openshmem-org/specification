\apisummary{
    Start a communication session.
}

\begin{apidefinition}

\begin{Csynopsis}
void @\FuncDecl{shmem\_session\_start}@(long options, shmem_ctx_t ctx);
\end{Csynopsis}

\begin{apiarguments}
    \apiargument{IN}{ctx}{A context handle specifying the context associated
    with this session.}
    \apiargument{IN}{options}{The set of requested options from
    Table~\ref{session_opts} for this session.  Multiple options may be
    requested by combining them with a bitwise OR operation; otherwise,
    \CONST{0} can be given if no options are requested.}
\end{apiarguments}

\apidescription{
    \FUNC{shmem\_session\_start} is a non-collective routine that begins a
    session on communication context \VAR{ctx} with hints requested via
    \VAR{options}.
    Sessions on a communication context must be stopped with a call to
    \FUNC{shmem\_session\_stop} on the same context.
    If a session is already started on a given context, another call to
    \FUNC{shmem\_session\_start} on that same context combines new options via a
    bitwise OR operation.
    Passing false or ambiguous \VAR{options} to a session should never result in
    undefined behavior, but may result in the library aborting the program.
}

\apireturnvalues{
    None.
}

\begin{longtable}{|p{0.45\textwidth}|p{0.5\textwidth}|}
    \hline
    \hline
    \textbf{Option} & \textbf{Usage hint}
    \tabularnewline \hline
    \endhead
    %%
    \LibConstDecl{SHMEM\_SESSION\_OP\_PUT} &
    \newline 
    The session will contain non-blocking \textit{put} and/or scalar put operations.
    \tabularnewline \hline

    \LibConstDecl{SHMEM\_SESSION\_OP\_GET} &
    \newline
    The session will contain non-blocking \textit{get} operations.
    \tabularnewline \hline

    \LibConstDecl{SHMEM\_SESSION\_OP\_PUT\_SIGNAL} &
    \newline
    The session will contain non-blocking \textit{put-with-signal} operations.
    \tabularnewline \hline
    
    \LibConstDecl{SHMEM\_SESSION\_OP\_AMO} &
    \newline
    The session will contain non-fetching AMOs.
    \tabularnewline \hline

    \LibConstDecl{SHMEM\_SESSION\_OP\_AMO\_FETCH} &
    \newline
    The session will contain non-blocking fetching AMOs.
    \tabularnewline \hline

    \LibConstDecl{SHMEM\_SESSION\_CHAIN} &
    \newline
    The session will contain a chain (a trivial repeating pattern) of similar RMA operations.
    \tabularnewline \hline 

    \LibConstDecl{SHMEM\_SESSION\_UNIFORM\_AMO} &
    \newline
    The session will contain a chain of AMOs that will not occur concurrently
    across any different signal operators (i.e.~\ref{subsec:signal_operator}),
    operations (\ref{sec:amo}), or types (Tables \ref{stdamotypes} and
    \ref{extamotypes}).
    \tabularnewline \hline
    \TableCaptionRef{Session options}
    \label{session_opts}
\end{longtable}

\apinotes{
    The \FUNC{shmem\_session\_start} routine provides hints for improving
    performance, and \openshmem implementations are not required to apply any
    optimization.
    \FUNC{shmem\_session\_start} is non-collective, so there is no implied
    synchronization.
    Implementations are encouraged to supply users with information about the
    session options being applied or ignored; for instance, when
    \LibConstRef{SHMEM\_DEBUG} is set.
}

\end{apidefinition}
