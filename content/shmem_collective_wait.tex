\apisummary{
    The routine waits until a collective operation identified by a request
    object completes.
}

\begin{apidefinition}

\begin{Csynopsis}
int @\FuncDecl{shmem\_collective\_wait}@(shmem_req_h request);
\end{Csynopsis}

\begin{apiarguments}

  \apiargument{IN}{request}{Request representing a outstanding collective}

\end{apiarguments}

\apidescription{

The \FUNC{shmem\_collective\_wait} function is a blocking operation. It is used to
determine whether a collective operation identified by the request object has been
completed. If the collective operation is completed,
\FUNC{shmem\_collective\_wait}
returns zero and deallocates the request object. If the collective operation has
not been completed, \FUNC{shmem\_collective\_wait} blocks until collective
operation completes and then returns zero.


In a multithreaded environment, \FUNC{shmem\_collective\_wait} can be called by different
threads but on different request objects. It is the responsibility of the
OpenSHMEM user to ensure that proper synchronization is used to prevent race
conditions or deadlock. Specifically, the \FUNC{shmem\_collective\_wait} operation should
be called after the collective operation to ensure that the request object is
not deallocated prematurely.
    }

\apireturnvalues{
    On success returns zero, otherwise returns a negative integer.
    }

\end{apidefinition}
