\apisummary{
    The routine waits until a operation identified by a request
    object completes.
}

\begin{apidefinition}

\begin{Csynopsis}
int @\FuncDecl{shmem\_req\_wait}@(shmem_req_h request);
\end{Csynopsis}

\begin{apiarguments}

  \apiargument{IN}{request}{Request handle}

\end{apiarguments}

\apidescription{

The \FUNC{shmem\_req\_wait} function is a blocking operation used to determine whether an
operation identified by the request object has been completed. If the operation
is completed, \FUNC{shmem\_req\_wait} returns zero and deallocates the request object. If
the operation has not been completed, \FUNC{shmem\_req\_wait} blocks until the operation
completes and then returns zero.


In a multithreaded environment, \FUNC{shmem\_req\_wait} can be called by different
threads but on different request objects. It is the responsibility of the
\openshmem user to ensure that proper synchronization is used to prevent race
conditions or deadlock. Specifically, the \FUNC{shmem\_req\_wait} operation should
be called after the collective operation to ensure that the request object is
not deallocated prematurely.
    }

\apireturnvalues{
    On success returns zero, otherwise returns a negative integer.
    }

\end{apidefinition}
