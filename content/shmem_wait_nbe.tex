\apisummary{
    The request wait routine provides a method for completing
    outstanding operations on a request.
}

\begin{apidefinition}

\begin{Csynopsis}
void shmem_wait_nbe(shmem_request_t *req);
\end{Csynopsis}

\begin{apiarguments}
    \apiargument{IN}{req}{Handle to the request object.}
\end{apiarguments}

\apidescription{
    The \FUNC{shmem\_wait\_nbe} routine blocks until the operation associated with the request is completed.
    There are no guarantees on the ordering of other outstanding
    operations (i.e. if two operations are posted, the completion of the second request using \FUNC{shmem\_wait\_nbe} or \FUNC{shmem\_test\_nbe} does not imply that the first operation is also completed).
}

\apireturnvalues{
    None.
}

\apinotes{ In a threaded environment calling \FUNC{shmem\_wait\_nbe} from
a thread other than the thread, posting the operation is permitted.
However, only one thread must call \FUNC{shmem\_wait\_nbe} for a
particular request. The \FUNC{shmem\_wait\_nbe} will return after the
operation is finished. It is the users responsibility that no new
operation is posted to the request while it is waited on. This may
require the use of a synchronization operation provided by the threading
package.
}

\end{apidefinition}
