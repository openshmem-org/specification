\apisummary{
    Registers the arrival of a \ac{PE} at a barrier and returns immediately. It completes when all \acp{PE}
    arrive at the barrier and all local updates and remote memory updates on the default context are completed. 
    }

\begin{apidefinition}

\begin{Csynopsis}
void @\FuncDecl{shmem\_barrier\_all\_nb}@(void);
\end{Csynopsis}

\begin{apiarguments}

    \apiargument{None.}{}{}

\end{apiarguments}

\apidescription{   
    Similar to the \FUNC{shmem\_barrier\_all} routine, the nonblocking \FUNC{shmem\_barrier\_all\_nb}
    is a mechanism for synchronizing all \acp{PE} in the world team at
    once. This routine completes when all \acp{PE} have called
    \FUNC{shmem\_barrier\_all\_nb}. 

   A call to the nonblocking barrier routine posts the operation and returns
    immediately without necessarily completing the operation. Upon successful posting of the operation, 
    an opaque request handle is created and returned. The
    operation is completed after a call to \FUNC{shmem\_req\_test} or
    \FUNC{shmem\_req\_wait}. When the operation is complete, the request handle
    is deallocated and cannot be reused.

    Prior to completion, \FUNC{shmem\_barrier\_all\_nb}
    ensures completion of all previously issued memory stores and remote memory
    updates issued on the default context via \openshmem \acp{AMO} and
    \ac{RMA} routine calls such
    as \FUNC{shmem\_int\_add}, \FUNC{shmem\_put32},
    \FUNC{shmem\_put\_nbi}, and \FUNC{shmem\_get\_nbi}.
}

\apireturnvalues{
    None.
}

\apinotes{
}

\end{apidefinition}
