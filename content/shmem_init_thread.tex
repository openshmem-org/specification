\apisummary{
Initializes the \openshmem{} library, similar to SHMEM\_INIT(), and in addition
performs the initialization required for thread-safe invocation of \openshmem{} functions.}

\begin{apidefinition}

\begin{Csynopsis}
int shmem_init_thread(int requested, int *provided);
\end{Csynopsis}

\begin{Fsynopsis}  
INTEGER requested, provided
     
SHMEM_INIT_THREAD(required, provided)
\end{Fsynopsis}  

\begin{apiarguments}
\apiargument{IN}{requested}  {The thread level support requested by the user.
The correct values are SHMEM\_THREAD\_SINGLE or SHMEM\_THREAD\_MULTIPLE}
\apiargument{OUT}{provided}{The thread level support provided by the \openshmem{} implementation.}
\end{apiarguments}


\apidescription{
    \FUNC{shmem\_thread\_init}  initializes the \openshmem{} library similar to
    \FUNC{shmem\_init}, and in addition performs the initialization required for
    thread-safe invocation of \openshmem{} functions. The argument
    \textit{requested} is used to specify the desired level of thread support.
    The function returns the support provided by the library. There are two
    hierarchical levels of threading support.
 
 \begin{itemize}
\item SHMEM\_THREAD\_SINGLE: One thread calls all \openshmem{} functions.
\openshmem{} implementers can assume there is no threading.

\item SHMEM\_THREAD\_FUNNELED: The application must ensure that only the main
thread makes the \openshmem{} calls. The \ac{PE} can either be
single-threaded or multi-threaded.
 
\item SHMEM\_THREAD\_MULTIPLE - Processes may have multiple threads. Any
thread may issue an \openshmem{} call at any time, subject to the semantics
described in this section.
\end{itemize}

The function may be used to initialize \openshmem{}, and to initialize the
\openshmem{} with thread safety, instead of SHMEM\_INIT.
}

\apireturnvalues{    
	NONE
    }

\apinotes{ 
The \openshmem{} programming model does not recognize individual threads.  Any
\openshmem{} operation initiated by a thread is considered an action of the
process as a whole. Thread-safety should not be activated unless needed.
Activating  thread-safety causes additional overhead even if no additional
threads are created or used.
}		

\apiimpnotes{

}

\end{apidefinition}

 
  
