\apisummary{
  A structure type representing team configuration arguments
}

\begin{apidefinition}

\begin{Csynopsis}
typedef struct {
  int no_collectives;
  int num_contexts;
} shmem_team_config_t;
\end{Csynopsis}

  \vspace{1.0em}

  \apidescription{
    A team configuration argument acts as both input and output to the
    \FUNC{shmem\_team\_split\_*} routines.
    As an input, it specifies the requested capabilities of the team to be
    created. Capabilities can be requested as either hints or requirements.

    If given configuration parameter input is a requirement, and the team creation
    cannot provide this capability, then team creation fails.
    If a given configuration parameter input is a hint, and the library
    succeeds in creating the team, the parameter will be updated to
    the actual configuration that the library was able to provide
    during team creation.

    The \VAR{no\_collectives} member allows for teams to be created
    without support for collective communications, which allows implementations
    to reduce team creation overheads for those teams.
    When its value is zero, it specifies that the team should have collective
    operations enabled.
    When nonzero, the team will not support collective operations, which
    allows implementations to reduce team creation overheads.

    The \VAR{num\_contexts} member specifies the total number of contexts
    created from this team that can simultaneously exist. These contexts
    may be created in any number of threads. A program
    may destroy any number of contexts made from this team and make
    any number of new ones so long as the total existing at any point
    remains less than \VAR{num\_contexts}.
    See Section~\ref{sec:ctx} for more on communication contexts and
    Section~\ref{subsec:shmem_team_create_ctx} for team-based context creation.

    When using the configuration structure to create teams, a mask parameter
    controls which fields to use and whether they are hints or requirements.
    Any configuration parameter that is not indicated in the mask will be
    ignored.
    So, a program does not have to set all fields in the config struct;
    only those for which it does not want the default values.

    A configuration mask value is created by combining individual field
    masks with through a bitwise OR operation of the following library constants:
    
  {
  \apitablerow{\LibConstRef{SHMEM\_TEAM\_NO\_COLLECTIVES}}{
    The team should be created using the value of the
    \VAR{no\_collectives} member of the configuration parameter
    \VAR{config} as a requirement.
  }
  \apitablerow{\LibConstRef{SHMEM\_TEAM\_NO\_COLLECTIVES\_HINT}}{
    The team should be created using the value of the
    \VAR{no\_collectives} member of the configuration parameter
    \VAR{config} as a hint.
  }
  \apitablerow{\LibConstRef{SHMEM\_TEAM\_NUM\_CONTEXTS}}{
    The team should be created using the value of the
    \VAR{num\_contexts} member of the configuration parameter
    \VAR{config} as a requirement.
  }
  \apitablerow{\LibConstRef{SHMEM\_TEAM\_NUM\_CONTEXTS\_HINT}}{
    The team should be created using the value of the
    \VAR{num\_contexts} member of the configuration parameter
    \VAR{config} as a hint.
  }  
  }

  If a program creates a mask using both the requirement and hint flag
  for a given parameter, the behavior is undefined.

   A configuration mask value of \CONST{0} indicates that the team
   should be created with the default values for all configuration
   parameters, as follows:

  {
  \apitablerow{no\_collectives = \CONST{0}}{
    By default, teams support collective operations
    }
  \apitablerow{num\_contexts = \CONST{0}}{
    By default, no contexts can be created on a new team
    }
  }

  }

  \apinotes{
    The library can set a team configuration to indicate that collective operations are
    disabled without providing some optimized version of teams without support for
    collective operations. Disabling collectives is a portable way to
    provide optimizations on some implementations, not a promise that teams with
    this option will reduce overheads on all implementations.
  }

\end{apidefinition}
