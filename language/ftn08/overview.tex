\openshmem \Fortran[bind(C)] language bindings is based on \Fortran-\Cstd
interoperability feature introduced in \Fortran[2018](ISO/IEC 1539-1:2018(E))
language standard. \Fortran[2018] defines a standard way to generate procedure,
derived-type declarations and global variables which are interoperable with
\Cstd(ISO/IEC 9899:2011). The bind(C) attribute available in the \Fortran
language enables the interoperability with \Cstd. The \openshmem
\Fortran[bind(C)] language bindings use this standard way to generate \Fortran
wrappers over the primary \Cstd language bindings provided by \openshmem.

\begin{itemize}
    \item All the \openshmem APIs specified by the \openshmem \Fortran[bind(C)]
    language bindings are implemented as a wrapper interface using \Fortran
    bind(C) attribute over the \Cstd language bindings specified by \openshmem
    specification.
    \item All available language constants, library handles, and environment
    variables from the \openshmem specification are supported by \openshmem
    \Fortran[bind(C)] language bindings.
    \item The \Fortran[bind(C)] language bindings is defined in a module named
    \textit{shmem}. The \textit{shmem} module must contain only the interfaces
    and constant names defined in this specification.
    \item All \openshmem extension APIs that are not part of this specification
    must be defined in a separate \textit{shmemx} module. The \textit{shmemx}
    module must exist, even if no extensions are provided. Any extensions shall
    use the \textit{shmemx\_} prefix for all routine, variable, and constant
    names.
    \item \openshmem APIs in this language bindings are available either as a
    subroutine or a function. \openshmem \Cstd APIs which returns a value are
    wrapped as functions, while those APIs that have some arguments and perform
    a specific operation without returning any values are wrapped as
    subroutines.
\end{itemize}
